% arara: xelatex
\documentclass[12pt]{article}

% \usepackage{physics}

\usepackage{hyperref}
\hypersetup{
    colorlinks=true,
    linkcolor=blue,
    filecolor=magenta,      
    urlcolor=cyan,
    pdftitle={Overleaf Example},
    pdfpagemode=FullScreen,
    }
\urlstyle{same}

\usepackage{tikzducks}

\usepackage{tikz} % картинки в tikz
\usepackage{microtype} % свешивание пунктуации

\usepackage{array} % для столбцов фиксированной ширины

\usepackage{indentfirst} % отступ в первом параграфе

\usepackage{sectsty} % для центрирования названий частей
\allsectionsfont{\centering}

\usepackage{amsmath, amsfonts, amssymb} % куча стандартных математических плюшек

\usepackage{mathtools}
\usepackage{comment}

\usepackage[top=2cm, left=1.2cm, right=1.2cm, bottom=2cm]{geometry} % размер текста на странице

\usepackage{lastpage} % чтобы узнать номер последней страницы

\usepackage{enumitem} % дополнительные плюшки для списков
%  например \begin{enumerate}[resume] позволяет продолжить нумерацию в новом списке
\usepackage{caption}

\usepackage{url} % to use \url{link to web}


\newcommand{\smallduck}{\begin{tikzpicture}[scale=0.3]
    \duck[
        cape=black,
        hat=black,
        mask=black
    ]
    \end{tikzpicture}}

\usepackage{fancyhdr} % весёлые колонтитулы
\pagestyle{fancy}
\lhead{КНАД, статистика}
\chead{}
\rhead{2024-04-01}
\lfoot{}
\cfoot{}
\rfoot{}

\renewcommand{\headrulewidth}{0.4pt}
\renewcommand{\footrulewidth}{0.4pt}

\usepackage{tcolorbox} % рамочки!

\usepackage{todonotes} % для вставки в документ заметок о том, что осталось сделать
% \todo{Здесь надо коэффициенты исправить}
% \missingfigure{Здесь будет Последний день Помпеи}
% \listoftodos - печатает все поставленные \todo'шки


% более красивые таблицы
\usepackage{booktabs}
% заповеди из докупентации:
% 1. Не используйте вертикальные линни
% 2. Не используйте двойные линии
% 3. Единицы измерения - в шапку таблицы
% 4. Не сокращайте .1 вместо 0.1
% 5. Повторяющееся значение повторяйте, а не говорите "то же"


\setcounter{MaxMatrixCols}{20}
% by crazy default pmatrix supports only 10 cols :)


\usepackage{fontspec}
\usepackage{libertine}
\usepackage{polyglossia}

\setmainlanguage{russian}
\setotherlanguages{english}

% download "Linux Libertine" fonts:
% http://www.linuxlibertine.org/index.php?id=91&L=1
% \setmainfont{Linux Libertine O} % or Helvetica, Arial, Cambria
% why do we need \newfontfamily:
% http://tex.stackexchange.com/questions/91507/
% \newfontfamily{\cyrillicfonttt}{Linux Libertine O}

\AddEnumerateCounter{\asbuk}{\russian@alph}{щ} % для списков с русскими буквами
\setlist[enumerate, 2]{label=\asbuk*),ref=\asbuk*}

%% эконометрические сокращения
\DeclareMathOperator{\Cov}{\mathbb{C}ov}
\DeclareMathOperator{\Corr}{\mathbb{C}orr}
\DeclareMathOperator{\Var}{\mathbb{V}ar}
\DeclareMathOperator{\pCorr}{\mathrm{pCorr}}
\DeclareMathOperator{\col}{col}
\DeclareMathOperator{\row}{row}

\let\P\relax
\DeclareMathOperator{\P}{\mathbb{P}}

\DeclarePairedDelimiter{\abs}{\lvert}{\rvert}
\DeclarePairedDelimiter{\scalp}{\langle}{\rangle}

\let\H\relax
\DeclareMathOperator{\H}{\mathbb{H}}
\DeclareMathOperator{\plim}{plim}

\DeclareMathOperator{\E}{\mathbb{E}}
% \DeclareMathOperator{\tr}{trace}
\DeclareMathOperator{\card}{card}

\DeclareMathOperator{\Convex}{Convex}

\newcommand \cN{\mathcal{N}}
\newcommand \dN{\mathcal{N}}


\newcommand \RR{\mathbb{R}}
\newcommand \NN{\mathbb{N}}

\newcommand{\dBern}{\mathrm{Bern}}
\newcommand{\dBin}{\mathrm{Bin}}
\newcommand{\dGamma}{\mathrm{Gamma}}
\newcommand{\dBeta}{\mathrm{Beta}}



\begin{document}


\begin{enumerate}
    \item Время до прихода автобуса в $i$-й день — случайная величина $x_i$, имеющая экспоненциальное распределение с неизвестной интенсивностью $\lambda$. 
    Величины $x_i$ независимы. 
    Если автобус не приходит за $10$ минут, то я ухожу с остановки и иду пешком. 
    Вектор $y$ содержит время, проведённое мной на остановке в каждый день.
    
    \begin{enumerate}
      \item {[4]} Найдите оценку $\lambda$ методом максимального правдоподобия по выборке $y = (5, 10, 6, 10)$.
      \item {[4]} Найдите оценку $\lambda$ методом максимального правдоподобия по 
      произвольной выборке $y$. 
      \item {[2]} Оцените методом максимального правдоподобия вероятность того, что я иду пешком по произвольной выборке $y$.
    \end{enumerate}
    
    
    \item Величины $(x_i)$ независимы и равномерно распределены на отрезке $[-a, 2a]$ с неизвестным $a > 0$.
    
    Я наблюдаю величины $y_i = \abs{x_i}$.

    \begin{enumerate}
        \item {[4]} Найдите $\E(y_i)$ и $\E(y_i^2)$.
        \item {[3]} Постройте оценку метода моментов параметра $a$ для произвольной выборки $y_1$, $y_2$, \dots, $y_n$, 
используя первый момент. 
\item {[3]} Постройте оценку метода моментов параметра $a$ для произвольной выборки $y_1$, $y_2$, \dots, $y_n$, 
используя второй момент. 
    \end{enumerate}

    \item Величины $(X_i)$ независимы и экспоненциально распределены с интенсивностью $\lambda$.
    
    Рассмотрим оценку неизвестного параметра $a = 1/\lambda^2$:
    \[
 \hat a = \frac{X_1^2 + X_2^2 + \dots + X_n^2}{2n + 1}
    \]

    \begin{enumerate}
        \item {[5]} Является ли оценка несмещённой?
        \item {[5]} Является ли оценка состоятельной?
    \end{enumerate}

\item Величины $(X_i)$ независимы и одинаково распределены с ожиданием $2a$ и дисперсией $a^2$.

Известно, что оценка $\hat b = (4 + 3\bar X)/(3 + \bar X)$ является состоятельной для параметра $b$. 
По выборке из 1000 наблюдений оказалось, что $\bar X = 2$.

\begin{enumerate}
    \item {[7]} Найдите стандартную ошибку $se(\hat b)$ с помощью дельта-метода.
    \item {[3]} Постройте 95\% асимптотический доверительный интервал для $b$.
\end{enumerate}

\item Илон Маск оценивает один неизвестный параметр $a$ методом максимального правдоподобия.
По выборке из $1000$ наблюдений оказалось, что $\hat a = 12$, а вторая производная лог-правдоподобия равна $\ell''(\hat a) = -400$.
\begin{enumerate}
    \item {[5]} Оцените информацию Фишера. 
    \item {[3]} Постройте 95\% асимптотический доверительный интервал для $a$.
    \item {[2]} Постройте 95\% асимптотический доверительный интервал для $a^3$ любым способом.
\end{enumerate}

\item Среди 100 случайно выбранных рептилоидов 20 любят вышки 5G. 
Постройте асимпотический 95\%-й доверительный интервал для доли рептилоидов, любящих вышки 5G, двумя способами:
\begin{enumerate}
    \item {[5]} Используя статистику $(\hat p - p)/\sqrt{\widehat{\Var}(\hat p)}$ с решением линейного неравенства.
    \item {[5]} Используя статистику $(\hat p - p)/\sqrt{\Var(\hat p)}$ с решением квадратного неравенства.
\end{enumerate}


\end{enumerate}

\end{document}


