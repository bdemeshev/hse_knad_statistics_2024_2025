% arara: xelatex
\documentclass[12pt]{article}

% \usepackage{physics}

\usepackage{hyperref}
\hypersetup{
    colorlinks=true,
    linkcolor=blue,
    filecolor=magenta,      
    urlcolor=cyan,
    pdftitle={Overleaf Example},
    pdfpagemode=FullScreen,
    }
\urlstyle{same}

\usepackage{tikzducks}

\usepackage{tikz} % картинки в tikz
\usepackage{microtype} % свешивание пунктуации

\usepackage{array} % для столбцов фиксированной ширины

\usepackage{indentfirst} % отступ в первом параграфе

\usepackage{sectsty} % для центрирования названий частей
\allsectionsfont{\centering}

\usepackage{amsmath, amsfonts, amssymb} % куча стандартных математических плюшек

\usepackage{mathtools}
\usepackage{comment}

\usepackage[top=2cm, left=1.2cm, right=1.2cm, bottom=2cm]{geometry} % размер текста на странице

\usepackage{lastpage} % чтобы узнать номер последней страницы

\usepackage{enumitem} % дополнительные плюшки для списков
%  например \begin{enumerate}[resume] позволяет продолжить нумерацию в новом списке
\usepackage{caption}

\usepackage{url} % to use \url{link to web}


\newcommand{\smallduck}{\begin{tikzpicture}[scale=0.3]
    \duck[
        cape=black,
        hat=black,
        mask=black
    ]
    \end{tikzpicture}}

\usepackage{fancyhdr} % весёлые колонтитулы
\pagestyle{fancy}
\lhead{}
\chead{}
\rhead{Демо версии на кр 2025}
\lfoot{}
\cfoot{}
\rfoot{}

\renewcommand{\headrulewidth}{0.4pt}
\renewcommand{\footrulewidth}{0.4pt}

\usepackage{tcolorbox} % рамочки!

\usepackage{todonotes} % для вставки в документ заметок о том, что осталось сделать
% \todo{Здесь надо коэффициенты исправить}
% \missingfigure{Здесь будет Последний день Помпеи}
% \listoftodos - печатает все поставленные \todo'шки


% более красивые таблицы
\usepackage{booktabs}
% заповеди из докупентации:
% 1. Не используйте вертикальные линни
% 2. Не используйте двойные линии
% 3. Единицы измерения - в шапку таблицы
% 4. Не сокращайте .1 вместо 0.1
% 5. Повторяющееся значение повторяйте, а не говорите "то же"


\setcounter{MaxMatrixCols}{20}
% by crazy default pmatrix supports only 10 cols :)


\usepackage{fontspec}
\usepackage{libertine}
\usepackage{polyglossia}

\setmainlanguage{russian}
\setotherlanguages{english}

% download "Linux Libertine" fonts:
% http://www.linuxlibertine.org/index.php?id=91&L=1
% \setmainfont{Linux Libertine O} % or Helvetica, Arial, Cambria
% why do we need \newfontfamily:
% http://tex.stackexchange.com/questions/91507/
% \newfontfamily{\cyrillicfonttt}{Linux Libertine O}

\AddEnumerateCounter{\asbuk}{\russian@alph}{щ} % для списков с русскими буквами
\setlist[enumerate, 2]{label=\asbuk*),ref=\asbuk*}

%% эконометрические сокращения
\DeclareMathOperator{\Cov}{\mathbb{C}ov}
\DeclareMathOperator{\Corr}{\mathbb{C}orr}
\DeclareMathOperator{\Var}{\mathbb{V}ar}
\DeclareMathOperator{\pCorr}{\mathrm{pCorr}}
\DeclareMathOperator{\col}{col}
\DeclareMathOperator{\row}{row}

\let\P\relax
\DeclareMathOperator{\P}{\mathbb{P}}

\DeclarePairedDelimiter{\abs}{\lvert}{\rvert}
\DeclarePairedDelimiter{\scalp}{\langle}{\rangle}

\let\H\relax
\DeclareMathOperator{\H}{\mathbb{H}}
\DeclareMathOperator{\plim}{plim}

\DeclareMathOperator{\E}{\mathbb{E}}
% \DeclareMathOperator{\tr}{trace}
\DeclareMathOperator{\card}{card}

\DeclareMathOperator{\Convex}{Convex}

\newcommand \cN{\mathcal{N}}
\newcommand \dN{\mathcal{N}}


\newcommand \RR{\mathbb{R}}
\newcommand \NN{\mathbb{N}}

\newcommand{\dBern}{\mathrm{Bern}}
\newcommand{\dBin}{\mathrm{Bin}}
\newcommand{\dGamma}{\mathrm{Gamma}}
\newcommand{\dBeta}{\mathrm{Beta}}



\begin{document}

\section*{Формат}

В контрольной будет 6 задач.
Задачи имеют равный вес. 
Продолжительность работы 120 минут. 
Можно будет использовать в качестве разрешенной шпаргалки один лист А4 со всех шести его сторон.



\section*{Вариант «Гоген»}
\begin{enumerate}
    \item Величины $(X_i)$ независимы и равномерно распределены на отрезке $[0, a]$, где $a > 0$.
    Я наблюдаю величину $X_i$, если $X_i > a/2$ и $0$, если $X_i \leq a/2$.
    \begin{enumerate}
        \item Постройте оценку максимального правдоподобия, если я пронаблюдал значения $5$, $6$, $0$.
        \item Постройте оценку максимального правдоподобия для произвольной выборки $y_1$, $y_2$, \dots, $y_n$.
    \end{enumerate}
    
    \item Величины $(X_i)$ независимы и равномерно распределены на отрезке $[0, a]$, где $a > 0$.
    Я наблюдаю величину $X_i$, если $X_i > a/2$ и $0$, если $X_i \leq a/2$.
    \begin{enumerate}
        \item Постройте оценку метода моментов, если я пронаблюдал значения $5$, $6$, $0$.
        \item Постройте оценку метода моментов для произвольной выборки $y_1$, $y_2$, \dots, $y_n$.
    \end{enumerate}

    \item Величины $(X_i)$ независимы и одинаково распределены с ожиданием $a$ и дисперсией $2a$.
    
    Рассмотрим оценку неизвестного $a$:
    \[
 \hat a = ((X_1 - X_2)^2 + (X_2 - X_3)^2 + \dots + (X_{n-1} - X_{n})^2) / 4n.
    \]

    \begin{enumerate}
        \item Является ли оценка несмещённой?
        \item Является ли оценка состоятельной?
    \end{enumerate}

\item Величины $(X_i)$ независимы и одинаково распределены с ожиданием $a$ и дисперсией $2a$.

Известно, что оценка $\hat b = (1 + \bar X)/(2 + \bar X)$ является состоятельной для параметра $b$. 
По выборке из 1000 наблюдений оказалось, что $\bar X = 2$.

\begin{enumerate}
    \item Найдите стандартную ошибку $se(\hat b)$ с помощью дельта-метода.
    \item Постройте 95\% асимптотический доверительный интервал для $b$.
\end{enumerate}

\item Илон Маск оценивает два параметра, $a$ и $b$, методом максимального правдоподобия. 
По выборке из $1000$ наблюдений оказалось, что $\hat a = 2$, $\hat b = 3$.
Матрица Гессе в точке максимума равна $H = \begin{pmatrix}
    -100 & 2 \\
    2 & -400 \\
\end{pmatrix}$.
\begin{enumerate}
    \item Оцените информацию Фишера. 
    \item Постройте 95\% асимптотический доверительный интервал для $a$.
    \item Постройте 95\% асимптотический доверительный интервал для $a - b$.
\end{enumerate}

\item Среди 500 рептилоидов 200 любят вышки 5G. 
Среди 700 жителей Нибиру вышки 5G любят 300 жителей. 

Постройте 99\%-й доверительный интервал для разницы долей рептилоидов и нибирутян, любящих вышки 5G.


\end{enumerate}

\end{document}

\section*{Вариант «Рафаэль»}

Скоро открытие :)

\begin{enumerate}
\item 
\end{enumerate}

\end{document}

%Из определения условного ожидания, $\E(X \mid A) = \E(X \cdot I_A) / \P(A)$, 
%легко получаются определения условной дисперсии, $\Var(X \mid A) = \E(X^2 \mid A) - (\E(X \mid A))^2$,
%условной ковариации $\Cov(X, Y \mid A) = \E(XY \mid A) - \E(X \mid A) \E(Y \mid A)$ и даже корреляции, 
%$\Corr(X, Y \mid A) = \Cov(X, Y \mid A) / \sqrt{\Var(X \mid A)\Var(Y \mid A)}$.

% здесь проектируемая часть



\end{document}

