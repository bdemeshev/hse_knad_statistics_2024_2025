% arara: xelatex
\documentclass[12pt]{article}

% \usepackage{physics}

\usepackage{hyperref}
\hypersetup{
    colorlinks=true,
    linkcolor=blue,
    filecolor=magenta,      
    urlcolor=cyan,
    pdftitle={Overleaf Example},
    pdfpagemode=FullScreen,
    }
\urlstyle{same}

\usepackage{tikzducks}

\usepackage{tikz} % картинки в tikz
\usepackage{microtype} % свешивание пунктуации

\usepackage{array} % для столбцов фиксированной ширины

\usepackage{indentfirst} % отступ в первом параграфе

\usepackage{sectsty} % для центрирования названий частей
\allsectionsfont{\centering}

\usepackage{amsmath, amsfonts, amssymb} % куча стандартных математических плюшек

\usepackage{mathtools}
\usepackage{comment}

\usepackage[top=2cm, left=1.2cm, right=1.2cm, bottom=2cm]{geometry} % размер текста на странице

\usepackage{lastpage} % чтобы узнать номер последней страницы

\usepackage{enumitem} % дополнительные плюшки для списков
%  например \begin{enumerate}[resume] позволяет продолжить нумерацию в новом списке
\usepackage{caption}

\usepackage{url} % to use \url{link to web}


\newcommand{\smallduck}{\begin{tikzpicture}[scale=0.3]
    \duck[
        cape=black,
        hat=black,
        mask=black
    ]
    \end{tikzpicture}}

\usepackage{fancyhdr} % весёлые колонтитулы
\pagestyle{fancy}
\lhead{}
\chead{}
\rhead{Контрольная работа, 17 мая 2025}
\lfoot{}
\cfoot{}
\rfoot{}

\renewcommand{\headrulewidth}{0.4pt}
\renewcommand{\footrulewidth}{0.4pt}

\usepackage{tcolorbox} % рамочки!

\usepackage{todonotes} % для вставки в документ заметок о том, что осталось сделать
% \todo{Здесь надо коэффициенты исправить}
% \missingfigure{Здесь будет Последний день Помпеи}
% \listoftodos - печатает все поставленные \todo'шки


% более красивые таблицы
\usepackage{booktabs}
% заповеди из докупентации:
% 1. Не используйте вертикальные линни
% 2. Не используйте двойные линии
% 3. Единицы измерения - в шапку таблицы
% 4. Не сокращайте .1 вместо 0.1
% 5. Повторяющееся значение повторяйте, а не говорите "то же"


\setcounter{MaxMatrixCols}{20}
% by crazy default pmatrix supports only 10 cols :)


\usepackage{fontspec}
\usepackage{libertine}
\usepackage{polyglossia}

\setmainlanguage{russian}
\setotherlanguages{english}

% download "Linux Libertine" fonts:
% http://www.linuxlibertine.org/index.php?id=91&L=1
% \setmainfont{Linux Libertine O} % or Helvetica, Arial, Cambria
% why do we need \newfontfamily:
% http://tex.stackexchange.com/questions/91507/
% \newfontfamily{\cyrillicfonttt}{Linux Libertine O}

\AddEnumerateCounter{\asbuk}{\russian@alph}{щ} % для списков с русскими буквами
\setlist[enumerate, 2]{label=\asbuk*),ref=\asbuk*}

%% эконометрические сокращения
\DeclareMathOperator{\Cov}{\mathbb{C}ov}
\DeclareMathOperator{\Corr}{\mathbb{C}orr}
\DeclareMathOperator{\Var}{\mathbb{V}ar}
\DeclareMathOperator{\pCorr}{\mathrm{pCorr}}
\DeclareMathOperator{\col}{col}
\DeclareMathOperator{\row}{row}

\let\P\relax
\DeclareMathOperator{\P}{\mathbb{P}}

\DeclarePairedDelimiter{\abs}{\lvert}{\rvert}
\DeclarePairedDelimiter{\scalp}{\langle}{\rangle}

\let\H\relax
\DeclareMathOperator{\H}{\mathbb{H}}
\DeclareMathOperator{\plim}{plim}

\DeclareMathOperator{\E}{\mathbb{E}}
% \DeclareMathOperator{\tr}{trace}
\DeclareMathOperator{\card}{card}

\DeclareMathOperator{\Convex}{Convex}

\newcommand \cN{\mathcal{N}}
\newcommand \dN{\mathcal{N}}


\newcommand \RR{\mathbb{R}}
\newcommand \NN{\mathbb{N}}

\newcommand{\dBern}{\mathrm{Bern}}
\newcommand{\dBin}{\mathrm{Bin}}
\newcommand{\dGamma}{\mathrm{Gamma}}
\newcommand{\dBeta}{\mathrm{Beta}}



\begin{document}

\begin{enumerate}
    \item В пруду встречаются караси, щуки и налимы. 
    Кот Матроскин поймал 200 рыб: 50 карасей, 70 щук и 80 налимов. 
    Кот Леопольд поймал 100 рыб: 10 карасей, 40 щук и 50 налимов. 

    \begin{enumerate}
        \item {[7]} С помощью критерия хи-квадрат Пирсона на уровне значимости $5\%$ проверьте гипотезу о независимости вида выловленной рыбы от рыбака.
        Альтернативная гипотеза состоит в том, что распределение видов рыбы зависит от рыбака.
        \item {[3]} Укажите $p$-значение для теста из пункта (а) с помощью подходящей функции распределения.
        Явно напишите, какая функция распределения используется. 
    \end{enumerate}

    Квантили уровня 95\% для $\chi^2$-распределение: $\chi^2_1 = 3.84$, $\chi^2_2 = 5.99$, $\chi^2_3 = 7.81$.


    \item Сёгун Минамото-но Ёритомо хочет отбирать на службу только опытных самураев. 
    Сила удара меча у опытного самурая равномерно распределена на отрезке $[3; 7]$,
    а у неопытного — равномерно на отрезке $[0; 4]$.

    Испытуемый самурай бьёт мечом и если сила удара оказалась больше порога $a$, то Ёримото принимает самурая на работу.

    Существует два типа ошибок. 
    Ошибка первого рода: на работу взяли неопытного самурая. 
    Ошибка второго рода: опытному самураю отказали в работе. 

    \begin{enumerate}
        \item {[3]} Найдите вероятности ошибок первого и второго рода при $a = 3.5$.
        \item {[7]} Постройте кривую зависимости ошибки второго рода от ошибки первого рода при различных порогах $a$.
    \end{enumerate}


    
    \item {[10]} Независимые величины $X_1$, $X_2$ распределены равномерно на отрезке $[0, a]$, где $a$ — неизвестный параметр.
    Мы наблюдаем только $X_1$.
    Постройте 95\%-й предиктивный интервал вида $[0, kX_1]$ для $X_2$. 


    \item Величины $(x_i)$ независимы и одинаково распределены с неизвестным ожиданием $\mu_x$ и конечной дисперсией.
    Величины $(y_i)$ независимы и одинаково распределены со своим неизвестным ожиданием $\mu_y$ и конечной дисперсией. 
    По независимым выборкам размера $n_x = 1000$, $n_y = 500$ оказалось, что $\bar x = 20$, $\bar y = 21$,
    а несмещённые выборочные дисперсии равны $200$ и $300$ соответственно. 

    Винни-Пух хочет протестировать гипотезу $H_0$: $\mu_x  = \mu_y$ против альтернативы $H_1$: $\mu_y > \mu_x$ на уровне значимости $2.5\%$.
    \begin{enumerate}
        \item {[7]} Проведите данные тест с помощью сравнения критического и наблюдаемого значения классической статистики. 
        \item {[3]} Укажите $p$-значение для теста из пункта (a) с помощью подходящей функции распределения.
    \end{enumerate}

    \item {[10]} Комиссар Жильбер измерил время приезда в минутах четырёх жёлтых такси, $x = (x_1, x_2, x_3, x_4)$, и четырёх белых такси $y = (y_1, y_2, y_3, y_4)$.
    Предположим, что законы распределения времени приезда непрерывны и могут отличаться только сдвигом на параметр $\mu$. 
    Наблюдения независимы.  Жильбер хочет проверить с помощью теста Манна-Уитни гипотезу $\mu = 0$ против альтернативы $\mu \neq 0$ на уровне значимости $5\%$.
    \begin{enumerate}
        \item {[4]} Чему равно максимально возможное значение статистики Манна Уитни в данном эксперименте? 
        Чему равна вероятность данного значения при верной $H_0$?
        \item {[6]} Проведите тест для выборок $x = (3.5, 6.5, 3.4, 2.8)$ и $y = (1.9, 2.5, 5.9, 1.6)$.
        Будет считать, что в данном случае корректно использовать нормальную аппроксимацию вместо точного закона распределения Манна-Уитни.
    \end{enumerate}
    
    
    

    \item Априорная функция плотности параметра $a$ равна $2a$ на отрезке $[0, 1]$ и нулю иначе.
    Наблюдаемая величина $X$ распределена равномерно на отрезке $[0, a]$ при фиксированном $a$.

    У нас есть единственное наблюдение, $X = 0.7$.

    \begin{enumerate}
        \item {[5]} Найдите апостериорное распределение параметра $a$. 
        \item {[3]} Найдите апостериорное ожидание и медиану параметра $a$. 
        \item {[2]} Постройте любой 94\% апостериорный интервал для $a$. 
    \end{enumerate}

\end{enumerate}

\end{document}



\end{document}

