% arara: xelatex
\documentclass[12pt]{article}

% \usepackage{physics}


\usepackage{hyperref}
\hypersetup{
    colorlinks=true,
    linkcolor=blue,
    filecolor=magenta,      
    urlcolor=cyan,
    pdftitle={Overleaf Example},
    pdfpagemode=FullScreen,
    }

\usepackage{tikzducks}

\usepackage{tikz} % картинки в tikz
\usepackage{microtype} % свешивание пунктуации

\usepackage{array} % для столбцов фиксированной ширины

\usepackage{indentfirst} % отступ в первом параграфе

\usepackage{sectsty} % для центрирования названий частей
\allsectionsfont{\centering}

\usepackage{amsmath, amsfonts, amssymb} % куча стандартных математических плюшек

\usepackage{comment}

\usepackage[top=2cm, left=1.2cm, right=1.2cm, bottom=2cm]{geometry} % размер текста на странице

\usepackage{lastpage} % чтобы узнать номер последней страницы

\usepackage{enumitem} % дополнительные плюшки для списков
%  например \begin{enumerate}[resume] позволяет продолжить нумерацию в новом списке
\usepackage{caption}

\usepackage{url} % to use \url{link to web}


\newcommand{\smallduck}{\begin{tikzpicture}[scale=0.3]
    \duck[
        cape=black,
        hat=black,
        mask=black
    ]
    \end{tikzpicture}}

\usepackage{fancyhdr} % весёлые колонтитулы
\pagestyle{fancy}
\lhead{Теория вероятностей и статистика: кнад}
\chead{}
\rhead{Домашние задания для самураев}
\lfoot{}
\cfoot{}
\rfoot{}

\renewcommand{\headrulewidth}{0.4pt}
\renewcommand{\footrulewidth}{0.4pt}

\usepackage{tcolorbox} % рамочки!

\usepackage{todonotes} % для вставки в документ заметок о том, что осталось сделать
% \todo{Здесь надо коэффициенты исправить}
% \missingfigure{Здесь будет Последний день Помпеи}
% \listoftodos - печатает все поставленные \todo'шки


% более красивые таблицы
\usepackage{booktabs}
% заповеди из докупентации:
% 1. Не используйте вертикальные линни
% 2. Не используйте двойные линии
% 3. Единицы измерения - в шапку таблицы
% 4. Не сокращайте .1 вместо 0.1
% 5. Повторяющееся значение повторяйте, а не говорите "то же"


\setcounter{MaxMatrixCols}{20}
% by crazy default pmatrix supports only 10 cols :)


\usepackage{fontspec}
\usepackage{libertine}
\usepackage{polyglossia}

\setmainlanguage{russian}
\setotherlanguages{english}

% download "Linux Libertine" fonts:
% http://www.linuxlibertine.org/index.php?id=91&L=1
% \setmainfont{Linux Libertine O} % or Helvetica, Arial, Cambria
% why do we need \newfontfamily:
% http://tex.stackexchange.com/questions/91507/
% \newfontfamily{\cyrillicfonttt}{Linux Libertine O}

\AddEnumerateCounter{\asbuk}{\russian@alph}{щ} % для списков с русскими буквами
\setlist[enumerate, 2]{label=\asbuk*),ref=\asbuk*}

%% эконометрические сокращения
\DeclareMathOperator{\Cov}{\mathbb{C}ov}
\DeclareMathOperator{\pCorr}{\mathrm{pCorr}}
\DeclareMathOperator{\Corr}{\mathbb{C}orr}
\DeclareMathOperator{\Var}{\mathbb{V}ar}
\DeclareMathOperator{\col}{col}
\DeclareMathOperator{\row}{row}

\let\P\relax
\DeclareMathOperator{\P}{\mathbb{P}}

\let\H\relax
\DeclareMathOperator{\H}{\mathbb{H}}

\DeclareMathOperator{\CE}{\mathrm{CE}}


\newcommand{\distto}{\overset{\text{dist}}{\to}}
\newcommand{\zcr}{z_{\text{cr}}}


\DeclareMathOperator{\E}{\mathbb{E}}
% \DeclareMathOperator{\tr}{trace}
\DeclareMathOperator{\card}{card}

\DeclareMathOperator{\Convex}{Convex}

\newcommand \cN{\mathcal{N}}
\newcommand \RR{\mathbb{R}}
\newcommand \NN{\mathbb{N}}

\newcommand{\dBin}{\mathrm{Bin}}




\begin{document}


\section*{Домашнее задание 1}

Дедлайн: 2025-02-04, 23:59.

\begin{enumerate}
    \item Случайные величины $y_i$ независимы и одинаково распределены с $\P(y_i = 0) = a$, $\P(y_i = 1) = 2a$, $\P(y_i = 2) = 1 - 3a$.
    В выборке $y_1$, $y_2$, \dots, $y_n$ оказалось $N_0$ нулей, $N_1$ единиц и $N_2$ двоек. 
    \begin{enumerate}
        \item Найдите оценку $\hat a$ параметра $a$ методом моментов используя $\E(y_i)$.
        \item Найдите оценку $\hat a$ параметра $a$ методом моментов используя $\E(y_i^2)$.
        \item Найдите оценку $\hat a$ параметра $a$ методом максимального правдоподобия.
    \end{enumerate}
    

\item Случайные величины $y_i$ независимы и нормально распределены $\cN(2a; a)$ с неизвестным параметром $a$.
\begin{enumerate}
    \item Найдите оценку $\hat a$ параметра $a$ методом моментов используя $\E(y_i)$.
    \item Найдите оценку $\hat a$ параметра $a$ методом моментов используя $\E(y_i^2)$.
    \item Найдите оценку $\hat a$ параметра $a$ методом максимального правдоподобия.
\end{enumerate}

\item В отделении банка 5 клиентских окошек. 
Время обслуживания каждого клиента имеет экспоненциальное распределение с неизвестной интенсивностью $\lambda$. 
Я был в очереди последним, и когда я встал к освободившемуся окошку номер 5, все остальные окошки ещё обслуживали клиентов. 
Через 3 минуты обслужили клиента в окошке 3, через 7 минут — клиента в окошке номер 4, 
а потом я освободился и ушёл. 

\begin{enumerate}
    \item Найдите оценку $\hat a$ параметра $a$ методом моментов, используя любое математическое ожидание. 
    \item Найдите оценку $\hat a$ параметра $a$ методом максимального правдоподобия.
\end{enumerate}

Примечание: если в данной задаче возникает нерешаемое в явном виде уравнение, то, конечно, можно и нужно воспользоваться подходящим численным методом. 

\end{enumerate}


\section*{Домашнее задание 2}

% свойства оценок
Дедлайн: 2025-02-23, 23:59.

Оцениваемые задачи:

\begin{enumerate}
    \item Величины $y_1$, $y_2$, \dots, $y_n$ независимы и равномерны отрезке на $[0; a]$ с неизвестным $a > 5$.
    Никола Тесла хочет оценить неизвестный параметр $b = \P(y_i > 5)$.

    Рассмотрим две оценки: $\hat b_n$ — доля наблюдений в выборке, оказавшихся больше $5$ и $\hat b'_n = 1 - 2.5/\bar y$.

    \begin{enumerate}
        \item Является ли оценка $\hat b_n$ несмещённой? состоятельной?
        \item Является ли оценка $\hat b'_n$ несмещённой? состоятельной?
    \end{enumerate}
    

\item Величины $y_i$ независимы и имеют функцию плотности
\[
f(y) = \begin{cases}
3y^2/\theta^3, \text{ если } y\in[0;\theta]; \\
0, \text{ иначе.}
\end{cases}
\]

\begin{enumerate}
\item Найдите оценку $\hat\theta$ неизвестного параметра $\theta$ методом максимального правдоподобия.
\item Является ли оценка $\hat\theta$ несмещённой? 
\item Является ли оценка $\hat\theta$ состоятельной?
\item Найдите функцию плотности оценки $\hat\theta$.
\item На какую величину нужно домножить оценку $\hat\theta$, чтобы она стала несмещённой?
\end{enumerate}

Подсказка: ответ на пункт (б) можно получить без вычислений и интегралов :)

\end{enumerate}

Неоцениваемые задачи в удовольствие:

\begin{enumerate}[resume]
\item Величина $Y$ имеет биномиальное распределение $\dBin(n, p)$.
\begin{enumerate}
\item Является ли оценка $\hat p = Y/n$ для $p$ несмещённой? Если является смещённой, то скорректируйте оценку так, чтобы она стала несмещённой.
\item Чему равна теоретическая дисперсия $\sigma^2$ величины $Y$?
\item Является ли оценка $\hat \sigma^2 = n \hat p (1- \hat p)$ для $\sigma^2$ несмещённой? Если является смещённой, то скорректируйте оценку так, чтобы она стала несмещённой.
\end{enumerate}

\item Величины $X_i$ независимы и одинаково распределены с неизвестными $\E(X_i) = \mu$ и $\Var(X_i) = \sigma^2$.

Рассмотрим четыре оценки:
\[
\hat \mu_A = (X_1 + X_2) / 2, \quad \hat\mu_B = (X_1 + X_2 + X_3) / 3,  \hat \mu_C = 2X_1 - X2, 
\quad \hat\mu_D = (X_1 + X_2 + \ldots +  X_{20}) / 21. 
\]

\begin{enumerate}
  \item Какая из приведенных оценок для $\mu$ является несмещённой?
  \item У какой несмещённой оценки самая маленькая дисперсия?
  \item Выберите наиболее эффективную оценку в этом множестве по критерию $MSE$, если $\sigma = 0.5\mu$. 
\end{enumerate}

\item Величины $X_1$ и $X_2$  независимы и равномерны на отрезке $[0;a]$ с неизвестным $a$ и $Y = \min\{ X_1, X_2\}$.

\begin{enumerate}
  \item При каком $\beta$ оценка $\hat a=\beta Y$ для параметра $a$ будет несмещённой?
  \item При каком $\beta$ оценка $\hat a=\beta Y$ для параметра $a$ будет наиболее эффективной по критерию $MSE$?
\end{enumerate}

\item Величины $X_i$ независимы и имеют закон распределения 

\begin{minipage}{\textwidth}
  \begin{tabular}{crrr}
    \toprule
     $x$ & $0$ & $1$ & $a$ \\
    \midrule
    $\P(X = x)$ & $1/4$ & $1/4$ & $2/4$ \\
    \bottomrule
\end{tabular}
\end{minipage}

\begin{enumerate}
    \item Постройте состоятельную оценку для неизвестного $a$.
    \item Возможно ли в этой задаче построить несмещённую оценку для $a$? 
\end{enumerate}


\end{enumerate}



\section*{Домашнее задание 3}

% интервальные оценки
Дедлайн: 2025-02-04, 23:59.

Оцениваемые задачи:

\begin{enumerate}
\item 
    

\item Апостерирное распределение параметра $\theta$ с учётом имеющихся наблюдений является треугольным на отрезке $[0; 40]$ с модой в точке $30$. 
    \begin{enumerate}
        \item Найдите апостериорное математическое ожидание $\theta$.
        \item Найдите апостериорную медиану $\theta$.
        \item Постройте 94\% байесовский интервал наивысшей плотности для $\theta$.
        \item Постройте 94\% симметричный по вероятности байесовский интервал для $\theta$.
    \end{enumerate}

Определение треугольного распределения можно найти, например, на википедии :)

\end{enumerate}

Неоцениваемые задачи в удовольствие:

\begin{enumerate}
    \item 

    \item Цыганка Роза ничего не понимает в статистике, 
    но у неё всегда с собой колода из 36 карт.
    
    Помогите цыганке Розе построить точный 95\%-й доверительный интервал для
    неизвестной вероятности $p$ того, что клиента ждёт дальняя дорога и казённый дом.
    
\end{enumerate}



% 
\section*{Домашнее задание 4}

% интервальные оценки
Дедлайн: 2025-05-18, 23:59.

Оцениваемые задачи:

\begin{enumerate}
\item Величины $(y_i)$ независимы и одинаково непрерывно распределены. 
Всего есть $1000$ наблюдений. 
Постройте 95\%-й интервал для 90\%-го квантиля с помощью выборочных квантилей. 

Если для вычисления необходимых выборочных квантилей использовался код, то приведите его. 
     
\item Есть две выборки: $x = (2.7, 3.5, 4.2, 6.7)$ и $y = (1.6, 2.9, 3.9)$.
Все наблюдения независимы. 
Величины $(x_i)$ одинаково непрерывно распределены между собой, величины $(y_i)$ одинаково непрерывно распределены между собой. 
Проверьте гипотезу $H_0$ об одинаковом законе распределения в двух выборках, против альтернативной $\P(x_i > y_j) > 0.5$
на уровне значимости $5\%$.

\begin{enumerate}
    \item Проведите тест Манна~— Уитни, используя точное распределение статистики.
    \item Проведите тест Манна~— Уитни, используя нормальную аппроксимацию. Укажите $p$-значение.
\end{enumerate}

\end{enumerate}

Неоцениваемые задачи в удовольствие:

\begin{enumerate}[resume]
    \item Рассмотрим тест знаковых рангов Уилкоксона и связанные пары наблюдений $(x_i, y_i)$.
    При верной $H_0$ разницы $D_i = x_i - y_i$ одинаково непрерывно распределены и независимы. 

    Рассмотрим сумму знаковых рангов $WSR = \sum_{i=1}^n \sign(D_i) \rank (\abs{D_i})$.

    Найдите ожидание $\E(WSR)$ и дисперсию $\Var(WSR)$ при верной $H_0$.

    \item Величины $(X_i)$ независимы и одинаково распределены с неизвестными $\E(X_i) = \mu$ и $\Var(X_i) = \sigma^2$.
    По выборке из $1000$ наблюдений оказалось, что $\bar X = 30$, а несмещённая выборочная дисперсия равна $900$. 
    \begin{enumerate}
        \item Постройте асимптотический 95\%-й доверительный интервал для $\mu$. 
        Укажите $p$-значение для гипотезы $H_0$: $\mu = 35$ против альтернативной $H_a$: $\mu \neq 35$.
        \item Постройте асимптотический 95\%-й предсказательный интервал для $X_{1001}$.
        \item Постройте асимптотический 95\%-й предсказательный интервал для $(X_{1001} + X_{1002})/2$.
    \end{enumerate}

    \item Бариста Борис заметил, что в последнее время посетители заказывают только капуччино и раф. 
    Предположим, что посетители выбирают напиток независимо друг от друга, 
    а вероятность выбора капуччино постоянна и равна неизвестному числу $p$.
    
    У Бориса есть только две гипотезы, $H_0: p =1/3$ и $H_a: p =2/3$, 
    в которые он до получения данных верит с вероятностями $0.6$ и $0.4$, соответственно.
    
    Из первых 100 утренних посетителей $S = 40$ выбрали капуччино. 
    Борис хочет измерить разными способами, насколько этот наблюдаемый результат соотносится с гипотезами.
    
    \begin{enumerate}
      \item Найдите $\P(H_0 \mid S=40)$ и $\P(H_a \mid S=40)$.
    \end{enumerate}
    
    Борис решил на следующий день повторить эксперимент и снова посчитать $S_{\text{new}}$, 
    количество клиентов из первых ста, которые выберут капуччино. 
    \begin{enumerate}[resume]
      \item Найдите $\P(S_{\text{new}} \geq S \mid S=40, H_0)$ и $\P(S_{\text{new}} \geq S \mid S=40, H_a)$.
    \end{enumerate}
    
    \begin{enumerate}[resume]
      \item Какие из вероятностей можно посчитать без мнения Бориса о $\P(H_0)$ и $\P(H_a)$?
      \item Какая из вероятностей называется $p$-значением\index{$p$-значение} для гипотезы $H_0$ и статистики $S$?
    \end{enumerate}
    
    
    \item По таблице сопряжённости проверьте гипотезу о независимости двух признаков на уровне значимости 5\%
    против альтернативной гипотезы о зависимости признаков. Укажите $p$-значение.
    
    \begin{tabular}{*{3}{c}}
        \toprule
         & $X = A$ & $X = B$ \\
        \midrule
        $Y = C$ & $50$ & $60$ \\
        $Y = D$ & $20$ & $30$ \\
        $Y = E$ & $60$ & $50$ \\
        \bottomrule
    \end{tabular}
        
    \item Рассмотрим таблицу сопряжённости
    
    \begin{tabular}{*{4}{c}}
        \toprule
        $X = A$ & $X = B$ & $X = C$ & $X = D$ \\
        \midrule
        $50$ & $70$ & $80$ & $60$ \\
        \bottomrule
    \end{tabular}
    
    \begin{enumerate}
        \item На уровне значимости 5\% проверьте гипотезу об одинаковых вероятностях $p_a = p_b = p_c = p_d$ против альтернативной о том, что хотя бы одно из равенств нарушено.
        \item На уровне значимости 5\% проверьте гипотезу об одинаковых вероятностях $p_a = p_b = p_c = p_d$ против альтернативной о том, что $p_a \neq p_b = p_c$.
        \item На уровне значимости 5\% проверьте гипотезу об одинаковых вероятностях $p_a = p_b = p_c$ против альтернативной о том, что $p_a \neq p_b = p_c$.
    \end{enumerate}

    В каждом случае укажите $p$-значение.
\end{enumerate}



% 

\section*{Домашнее задание 5}

% интервальные оценки
Дедлайн: 2025-06-01, 23:59.

Оцениваемые задачи:

\begin{enumerate}
\item Величины $(y_i)$ независимы и экспоненциально распределены с интенсивностью $\lambda$.

Количество наблюдений $n$ велико. 
Тестируем гипотезу $H_0$: $\lambda = 2$ против альтернативы $\lambda \neq 2$.
\begin{enumerate}
    \item Выведите формулы для теста отношения правдоподобия $LR$, теста множителей Лагранжа $LM$ и теста Вальда $W$.
    \item Проведите тесты для конкретной выборки с $n = 1000$, $\bar y = 2.2$ и уровня значимости 1\%.
\end{enumerate}
     
\item Величины $(y_i)$ независимы и нормально распределены $\cN(\mu, 1)$.

Количество наблюдений $n$ велико. 
Тестируем гипотезу $H_0$: $\mu = 0$ против альтернативы $\mu \neq 0$.
\begin{enumerate}
    \item Выведите формулы для теста отношения правдоподобия $LR$, теста множителей Лагранжа $LM$ и теста Вальда $W$.
    \item Проведите тесты для конкретной выборки с $n = 1000$, $\sum y_i = 1000$, $\sum y_i^2 = 4000$ и уровня значимости 1\%.
\end{enumerate}

\end{enumerate}

Неоцениваемые задачи в удовольствие:

\begin{enumerate}[resume]
    \item Гипотеза $H_0$ описывается $5$-ю независимыми уравнениями, неограниченный максимум лог-правдоподобия равен $\ell_{UR} = -200$, а ограниченный — $\ell_R=-209$.
    Число наблюдений $n$ велико. Альтернативная гипотеза состоит в том, что хотя бы одно уравнение не выполнено.

    \begin{enumerate}
        \item Отвергается ли $H_0$ на уровне значимости $1\%$?
        \item Найдите $p$-значение. 
    \end{enumerate}
    
    \item Оценка неизвестного вектора параметров $a = (a_1, a_2, a_3)$  равна $\hat a = (1, 2, 3)$ с оценкой ковариационной матрицы
    \[
    \widehat{\Var}(\hat a) = \begin{pmatrix}
        9 & -1 & 2 \\
         & 16 & -1 \\
         & & 10 \\
    \end{pmatrix}.
    \]
    Число наблюдений велико.
    Рассмотрим гипотезу $H_0$: $a_1 = a_2 = a_3$ против альтернативы о том, что хотя бы одно уравнение не выполнено.
    \begin{enumerate}
        \item Предложите естественную оценку $\hat b$ для вектора $b = (a_1 - a_2, a_2 - a_3)$.
        \item Оцените ковариационную матрицу $\widehat{\Var}(\hat b)$.
        \item Переформулируйте $H_0$ в терминах вектора $b$.
        \item Проведите тест Вальда гипотезы $H_0$ на уровне значимости $5\%$.
    \end{enumerate}
    
    \item Мы оцениваем три неизвестных параметра, $(\theta_1, \theta_2, \theta_3)$. 
    При максимизации с учётом ограничений гипотезы $H_0$ оказывается, что градиент лог-правдоподобия равен $\grad \ell = (-0.1, 0.2, 0)$,
    а матрица Гессе в точке ограниченного экстремума равна 
    \[
    H = \begin{pmatrix}
        -5 & -2 & 0 \\
         & -6 & 0 \\
         & & -10
    \end{pmatrix}
    \]
    Число наблюдений велико. 

    \begin{enumerate}
        \item Чему равен градиент лог-правдоподобия в точке неограниченного экстремума?
        \item Протестируйте $H_0$ на уровне значимости 1\% с помощью теста множителей Лагранжа.
    \end{enumerate}

    \item Вспомним классический хи-квадрат тест Пирсона на соответствие выборки заданному дискретному закону распределения со 
    статистикой 
    \[
    S = \sum_{i=1}^k \frac{(f_i - np_i)^2}{np_i},
    \]
    где $k$ — число клеток таблицы, $f_i$ — количество наблюдений, попавших в $i$-ую клетку таблицы, а $p_1$, $p_2$, \dots, $p_k$ — вероятности, предполагаемые в $H_0$.
    
    С каким тестом ($LR$/$LM$/$W$) совпадает данная статистика?
    
    
    \item 

    \item \end{enumerate}



% 

\section*{Домашнее задание 6}

% интервальные оценки
Дедлайн: 2025-06-18, 23:59.

Оцениваемые задачи:

\begin{enumerate}
\item Случайные величины $y_1$, \dots, $y_n$ независимы и одинаково распределены с $\P(y_i = 1) = p$ и $\P(y_i = 0) = 1 - p$. 
Рассмотрим бутстрэп-выборку $y_1^*$, \dots, $y_n^*$.
\begin{enumerate}
    \item Найдите $\P(y_1^* = y_1)$ и $\P(y_1^* = y_2^*)$.
    \item Найдите $\E(y_i^*)$ и $\Var(y_i^*)$.
    \item Найдите $\Cov(y_1^*, y_2^*)$.
\end{enumerate}

\item Винни-Пух хочет проверить $5$ нулевых гипотез. 
Он посчитал $p$-значения для каждой из них, $p = (0.03, 0.04, 0.08, 0.15, 0.30)$.
\begin{enumerate}
    \item Какие гипотезы отвергнет алгоритм Холма~— Бонферонни, гарантирующий $FWER = 0.2$?
    \item Какие гипотезы отвергнет алгоритм Беньямини~— Хохберга, гарантирующий $FDR = 0.2$?
\end{enumerate}
     

\end{enumerate}

Неоцениваемые задачи в удовольствие:

\begin{enumerate}[resume]
    \item Величины $y_1$, \dots, $y_n$ независимы и одинаково распределены 
    с функцией плотности 
    \[
    f(y) = \begin{cases}
        \exp(a - y) \text{ если } y \geq a, \\
        0, \text{ иначе.}
    \end{cases}
    \]
    \begin{enumerate}
        \item Найдите оценку неизвестного параметра $a$ методом максимального правдоподобия. 
        \item Какова фактическая вероятность накрытия параметра $a$ при построении наивного бутстрэп доверительного интервала с номинальной вероятностью накрытия $95\%$?
    \end{enumerate}

    \item Опишите алгоритм наивного бутстрэпа для построения 95\% доверительного интервала для истинной медианы распределения. 

    \item У исследователя всего две нулевых гипотезы. 
    Каждая из них априорно верно с вероятностью $0.2$ независимо от других. 
    При верной отдельной нулевой гипотезы $H_j$ распределение соответствующей ей тестовой статистики непрерывно.
    Для упрощения будем считать, что если отдельная нулевая гипотеза $H_j$ не верна, то её $p$-значение в точности равно $0$.

    \begin{enumerate}
        \item Вспомните, как распределено $p$-значение при верной нулевой гипотезе.
        \item Рассмотри алгоритм Холма~— Бонферонни, гарантирующий $FWER = 0.2$.
        Какова условная вероятность того, что он отвергнет конкретную нулевую гипотезу с известным $p$-значением равным $u$?
        \item Рассмотри алгоритм Беньямини~— Хохберга, гарантирующий $FDR = 0.2$.
        Какова условная вероятность того, что он отвергнет конкретную нулевую гипотезу с известным $p$-значением равным $u$?
    \end{enumerate}

    \item В одном из вариантов бутстрэпа (subsampling bootstrap) в бутстрэп выборку попадает $m < n$ наблюдений без повторов из исходной выборки в $n$ наблюдений. 
    Предположим, что исходные $n$ наблюдений $y_1$, $y_2$, \dots, $y_n$ независимы и равномерны $\dUnif[0, 1]$.
    Рассмотрим бутстрэп выборку $y_1^*$, \dots, $y_m^*$.
    \begin{enumerate}
        \item Как распределено $y_i^*$?
        \item Найдите $\Corr(y_1^*, y_2^* \mid y_1, y_2, \dots, y_n)$.
        \item Найдите $\Corr(y_1^*, y_2^*)$.
    \end{enumerate}

 \end{enumerate}



% \input{ha-07_body.tex}



\end{document}

% здесь проектируемая часть

\end{document}

