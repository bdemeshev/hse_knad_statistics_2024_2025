
\documentclass[12pt]{article}

% \usepackage{physics}

\usepackage{hyperref}
\hypersetup{
    colorlinks=true,
    linkcolor=blue,
    filecolor=magenta,      
    urlcolor=cyan,
    pdftitle={Overleaf Example},
    pdfpagemode=FullScreen,
    }

\usepackage{tikzducks}

\usepackage{tikz} % картинки в tikz
\usepackage{microtype} % свешивание пунктуации

\usepackage{array} % для столбцов фиксированной ширины

\usepackage{indentfirst} % отступ в первом параграфе

\usepackage{sectsty} % для центрирования названий частей
\allsectionsfont{\centering}

\usepackage{amsmath, amsfonts, amssymb} % куча стандартных математических плюшек

\usepackage{comment}

\usepackage[top=2cm, left=1.2cm, right=1.2cm, bottom=2cm]{geometry} % размер текста на странице

\usepackage{lastpage} % чтобы узнать номер последней страницы

\usepackage{enumitem} % дополнительные плюшки для списков
%  например \begin{enumerate}[resume] позволяет продолжить нумерацию в новом списке
\usepackage{caption}

\usepackage{url} % to use \url{link to web}


\newcommand{\smallduck}{\begin{tikzpicture}[scale=0.3]
    \duck[
        cape=black,
        hat=black,
        mask=black
    ]
    \end{tikzpicture}}

\usepackage{fancyhdr} % весёлые колонтитулы
\pagestyle{fancy}
\lhead{}
\chead{}
\rhead{Домашние задания для самураев}
\lfoot{}
\cfoot{}
\rfoot{}

\renewcommand{\headrulewidth}{0.4pt}
\renewcommand{\footrulewidth}{0.4pt}

\usepackage{tcolorbox} % рамочки!

\usepackage{todonotes} % для вставки в документ заметок о том, что осталось сделать
% \todo{Здесь надо коэффициенты исправить}
% \missingfigure{Здесь будет Последний день Помпеи}
% \listoftodos - печатает все поставленные \todo'шки


% более красивые таблицы
\usepackage{booktabs}
% заповеди из докупентации:
% 1. Не используйте вертикальные линни
% 2. Не используйте двойные линии
% 3. Единицы измерения - в шапку таблицы
% 4. Не сокращайте .1 вместо 0.1
% 5. Повторяющееся значение повторяйте, а не говорите "то же"


\setcounter{MaxMatrixCols}{20}
% by crazy default pmatrix supports only 10 cols :)


\usepackage{fontspec}
\usepackage{libertine}
\usepackage{polyglossia}

\setmainlanguage{russian}
\setotherlanguages{english}

% download "Linux Libertine" fonts:
% http://www.linuxlibertine.org/index.php?id=91&L=1
% \setmainfont{Linux Libertine O} % or Helvetica, Arial, Cambria
% why do we need \newfontfamily:
% http://tex.stackexchange.com/questions/91507/
% \newfontfamily{\cyrillicfonttt}{Linux Libertine O}

\AddEnumerateCounter{\asbuk}{\russian@alph}{щ} % для списков с русскими буквами
\setlist[enumerate, 2]{label=\asbuk*),ref=\asbuk*}

%% эконометрические сокращения
\DeclareMathOperator{\pCorr}{\mathrm{pCorr}}
\DeclareMathOperator{\Cov}{\mathbb{C}ov}
\DeclareMathOperator{\Corr}{\mathbb{C}orr}
\DeclareMathOperator{\Var}{\mathbb{V}ar}
\DeclareMathOperator{\col}{col}
\DeclareMathOperator{\row}{row}

\let\P\relax
\DeclareMathOperator{\P}{\mathbb{P}}

\DeclareMathOperator{\E}{\mathbb{E}}
% \DeclareMathOperator{\tr}{trace}
\DeclareMathOperator{\card}{card}

\DeclareMathOperator{\Convex}{Convex}

\newcommand \cN{\mathcal{N}}
\newcommand \dN{\mathcal{N}}
\newcommand \dBin{\mathrm{Bin}}


\newcommand \RR{\mathbb{R}}
\newcommand \NN{\mathbb{N}}





\begin{document}



\section*{Домашнее задание 5}

% интервальные оценки
Дедлайн: 2025-06-01, 23:59.

Оцениваемые задачи:

\begin{enumerate}
\item Величины $(y_i)$ независимы и экспоненциально распределены с интенсивностью $\lambda$.

Количество наблюдений $n$ велико. 
Тестируем гипотезу $H_0$: $\lambda = 2$ против альтернативы $\lambda \neq 2$.
\begin{enumerate}
    \item Выведите формулы для теста отношения правдоподобия $LR$, теста множителей Лагранжа $LM$ и теста Вальда $W$.
    \item Проведите тесты для конкретной выборки с $n = 1000$, $\bar y = 2.2$ и уровня значимости 1\%.
\end{enumerate}
     
\item Величины $(y_i)$ независимы и нормально распределены $\cN(\mu, 1)$.

Количество наблюдений $n$ велико. 
Тестируем гипотезу $H_0$: $\mu = 0$ против альтернативы $\mu \neq 0$.
\begin{enumerate}
    \item Выведите формулы для теста отношения правдоподобия $LR$, теста множителей Лагранжа $LM$ и теста Вальда $W$.
    \item Проведите тесты для конкретной выборки с $n = 1000$, $\sum y_i = 1000$, $\sum y_i^2 = 4000$ и уровня значимости 1\%.
\end{enumerate}

\end{enumerate}

Неоцениваемые задачи в удовольствие:

\begin{enumerate}[resume]
    \item Гипотеза $H_0$ описывается $5$-ю независимыми уравнениями, неограниченный максимум лог-правдоподобия равен $\ell_{UR} = -200$, а ограниченный — $\ell_R=-209$.
    Число наблюдений $n$ велико. Альтернативная гипотеза состоит в том, что хотя бы одно уравнение не выполнено.

    \begin{enumerate}
        \item Отвергается ли $H_0$ на уровне значимости $1\%$?
        \item Найдите $p$-значение. 
    \end{enumerate}
    
    \item Оценка неизвестного вектора параметров $a = (a_1, a_2, a_3)$  равна $\hat a = (1, 2, 3)$ с оценкой ковариационной матрицы
    \[
    \widehat{\Var}(\hat a) = \begin{pmatrix}
        9 & -1 & 2 \\
         & 16 & -1 \\
         & & 10 \\
    \end{pmatrix}.
    \]
    Число наблюдений велико.
    Рассмотрим гипотезу $H_0$: $a_1 = a_2 = a_3$ против альтернативы о том, что хотя бы одно уравнение не выполнено.
    \begin{enumerate}
        \item Предложите естественную оценку $\hat b$ для вектора $b = (a_1 - a_2, a_2 - a_3)$.
        \item Оцените ковариационную матрицу $\widehat{\Var}(\hat b)$.
        \item Переформулируйте $H_0$ в терминах вектора $b$.
        \item Проведите тест Вальда гипотезы $H_0$ на уровне значимости $5\%$.
    \end{enumerate}
    
    \item Мы оцениваем три неизвестных параметра, $(\theta_1, \theta_2, \theta_3)$. 
    При максимизации с учётом ограничений гипотезы $H_0$ оказывается, что градиент лог-правдоподобия равен $\grad \ell = (-0.1, 0.2, 0)$,
    а матрица Гессе в точке ограниченного экстремума равна 
    \[
    H = \begin{pmatrix}
        -5 & -2 & 0 \\
         & -6 & 0 \\
         & & -10
    \end{pmatrix}
    \]
    Число наблюдений велико. 

    \begin{enumerate}
        \item Чему равен градиент лог-правдоподобия в точке неограниченного экстремума?
        \item Протестируйте $H_0$ на уровне значимости 1\% с помощью теста множителей Лагранжа.
    \end{enumerate}

    \item Вспомним классический хи-квадрат тест Пирсона на соответствие выборки заданному дискретному закону распределения со 
    статистикой 
    \[
    S = \sum_{i=1}^k \frac{(f_i - np_i)^2}{np_i},
    \]
    где $k$ — число клеток таблицы, $f_i$ — количество наблюдений, попавших в $i$-ую клетку таблицы, а $p_1$, $p_2$, \dots, $p_k$ — вероятности, предполагаемые в $H_0$.
    
    С каким тестом ($LR$/$LM$/$W$) совпадает данная статистика?
    
    
    \item 

    \item \end{enumerate}



\end{document}
