\section*{Домашнее задание 1}

Дедлайн: 2025-02-04, 23:59.

\begin{enumerate}
    \item Случайные величины $y_i$ независимы и одинаково распределены с $\P(y_i = 0) = a$, $\P(y_i = 1) = 2a$, $\P(y_i = 2) = 1 - 3a$.
    В выборке $y_1$, $y_2$, \dots, $y_n$ оказалось $N_0$ нулей, $N_1$ единиц и $N_2$ двоек. 
    \begin{enumerate}
        \item Найдите оценку $\hat a$ параметра $a$ методом моментов используя $\E(y_i)$.
        \item Найдите оценку $\hat a$ параметра $a$ методом моментов используя $\E(y_i^2)$.
        \item Найдите оценку $\hat a$ параметра $a$ методом максимального правдоподобия.
    \end{enumerate}
    

\item Случайные величины $y_i$ независимы и нормально распределены $\cN(2a; a)$ с неизвестным параметром $a$.
\begin{enumerate}
    \item Найдите оценку $\hat a$ параметра $a$ методом моментов используя $\E(y_i)$.
    \item Найдите оценку $\hat a$ параметра $a$ методом моментов используя $\E(y_i^2)$.
    \item Найдите оценку $\hat a$ параметра $a$ методом максимального правдоподобия.
\end{enumerate}

\item В отделении банка 5 клиентских окошек. 
Время обслуживания каждого клиента имеет экспоненциальное распределение с неизвестной интенсивностью $\lambda$. 
Я был в очереди последним, и когда я встал к освободившемуся окошку номер 5, все остальные окошки ещё обслуживали клиентов. 
Через 3 минуты обслужили клиента в окошке 3, через 7 минут — клиента в окошке номер 4, 
а потом я освободился и ушёл. 

\begin{enumerate}
    \item Найдите оценку $\hat a$ параметра $a$ методом моментов, используя любое математическое ожидание. 
    \item Найдите оценку $\hat a$ параметра $a$ методом максимального правдоподобия.
\end{enumerate}

Примечание: если в данной задаче возникает нерешаемое в явном виде уравнение, то, конечно, можно и нужно воспользоваться подходящим численным методом. 

\end{enumerate}
