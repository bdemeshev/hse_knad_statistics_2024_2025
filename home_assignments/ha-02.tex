\documentclass[12pt]{article}

% \usepackage{physics}

\usepackage{hyperref}
\hypersetup{
    colorlinks=true,
    linkcolor=blue,
    filecolor=magenta,      
    urlcolor=cyan,
    pdftitle={Overleaf Example},
    pdfpagemode=FullScreen,
    }

\usepackage{tikzducks}

\usepackage{tikz} % картинки в tikz
\usepackage{microtype} % свешивание пунктуации

\usepackage{array} % для столбцов фиксированной ширины

\usepackage{indentfirst} % отступ в первом параграфе

\usepackage{sectsty} % для центрирования названий частей
\allsectionsfont{\centering}

\usepackage{amsmath, amsfonts, amssymb} % куча стандартных математических плюшек

\usepackage{comment}

\usepackage[top=2cm, left=1.2cm, right=1.2cm, bottom=2cm]{geometry} % размер текста на странице

\usepackage{lastpage} % чтобы узнать номер последней страницы

\usepackage{enumitem} % дополнительные плюшки для списков
%  например \begin{enumerate}[resume] позволяет продолжить нумерацию в новом списке
\usepackage{caption}

\usepackage{url} % to use \url{link to web}


\newcommand{\smallduck}{\begin{tikzpicture}[scale=0.3]
    \duck[
        cape=black,
        hat=black,
        mask=black
    ]
    \end{tikzpicture}}

\usepackage{fancyhdr} % весёлые колонтитулы
\pagestyle{fancy}
\lhead{}
\chead{}
\rhead{Домашние задания для самураев}
\lfoot{}
\cfoot{}
\rfoot{}

\renewcommand{\headrulewidth}{0.4pt}
\renewcommand{\footrulewidth}{0.4pt}

\usepackage{tcolorbox} % рамочки!

\usepackage{todonotes} % для вставки в документ заметок о том, что осталось сделать
% \todo{Здесь надо коэффициенты исправить}
% \missingfigure{Здесь будет Последний день Помпеи}
% \listoftodos - печатает все поставленные \todo'шки


% более красивые таблицы
\usepackage{booktabs}
% заповеди из докупентации:
% 1. Не используйте вертикальные линни
% 2. Не используйте двойные линии
% 3. Единицы измерения - в шапку таблицы
% 4. Не сокращайте .1 вместо 0.1
% 5. Повторяющееся значение повторяйте, а не говорите "то же"


\setcounter{MaxMatrixCols}{20}
% by crazy default pmatrix supports only 10 cols :)


\usepackage{fontspec}
\usepackage{libertine}
\usepackage{polyglossia}

\setmainlanguage{russian}
\setotherlanguages{english}

% download "Linux Libertine" fonts:
% http://www.linuxlibertine.org/index.php?id=91&L=1
% \setmainfont{Linux Libertine O} % or Helvetica, Arial, Cambria
% why do we need \newfontfamily:
% http://tex.stackexchange.com/questions/91507/
% \newfontfamily{\cyrillicfonttt}{Linux Libertine O}

\AddEnumerateCounter{\asbuk}{\russian@alph}{щ} % для списков с русскими буквами
\setlist[enumerate, 2]{label=\asbuk*),ref=\asbuk*}

%% эконометрические сокращения
\DeclareMathOperator{\Cov}{\mathbb{C}ov}
\DeclareMathOperator{\Corr}{\mathbb{C}orr}
\DeclareMathOperator{\Var}{\mathbb{V}ar}
\DeclareMathOperator{\col}{col}
\DeclareMathOperator{\row}{row}

\let\P\relax
\DeclareMathOperator{\P}{\mathbb{P}}

\DeclareMathOperator{\E}{\mathbb{E}}
% \DeclareMathOperator{\tr}{trace}
\DeclareMathOperator{\card}{card}

\DeclareMathOperator{\Convex}{Convex}

\newcommand \cN{\mathcal{N}}
\newcommand \dN{\mathcal{N}}
\newcommand \dBin{\mathrm{Bin}}


\newcommand \RR{\mathbb{R}}
\newcommand \NN{\mathbb{N}}





\begin{document}

\section*{Домашнее задание 2}

% свойства оценок
Дедлайн: 2025-02-23, 23:59.

Оцениваемые задачи:

\begin{enumerate}
    \item Величины $y_1$, $y_2$, \dots, $y_n$ независимы и равномерны отрезке на $[0; a]$ с неизвестным $a > 5$.
    Никола Тесла хочет оценить неизвестный параметр $b = \P(y_i > 5)$.

    Рассмотрим две оценки: $\hat b_n$ — доля наблюдений в выборке, оказавшихся больше $5$ и $\hat b'_n = 1 - 2.5/\bar y$.

    \begin{enumerate}
        \item Является ли оценка $\hat b_n$ несмещённой? состоятельной?
        \item Является ли оценка $\hat b'_n$ несмещённой? состоятельной?
    \end{enumerate}
    

\item Величины $y_i$ независимы и имеют функцию плотности
\[
f(y) = \begin{cases}
3y^2/\theta^3, \text{ если } y\in[0;\theta]; \\
0, \text{ иначе.}
\end{cases}
\]

\begin{enumerate}
\item Найдите оценку $\hat\theta$ неизвестного параметра $\theta$ методом максимального правдоподобия.
\item Является ли оценка $\hat\theta$ несмещённой? 
\item Является ли оценка $\hat\theta$ состоятельной?
\item Найдите функцию плотности оценки $\hat\theta$.
\item На какую величину нужно домножить оценку $\hat\theta$, чтобы она стала несмещённой?
\end{enumerate}

Подсказка: ответ на пункт (б) можно получить без вычислений и интегралов :)

\end{enumerate}

Неоцениваемые задачи в удовольствие:

\begin{enumerate}[resume]
\item Величина $Y$ имеет биномиальное распределение $\dBin(n, p)$.
\begin{enumerate}
\item Является ли оценка $\hat p = Y/n$ для $p$ несмещённой? Если является смещённой, то скорректируйте оценку так, чтобы она стала несмещённой.
\item Чему равна теоретическая дисперсия $\sigma^2$ величины $Y$?
\item Является ли оценка $\hat \sigma^2 = n \hat p (1- \hat p)$ для $\sigma^2$ несмещённой? Если является смещённой, то скорректируйте оценку так, чтобы она стала несмещённой.
\end{enumerate}

\item Величины $X_i$ независимы и одинаково распределены с неизвестными $\E(X_i) = \mu$ и $\Var(X_i) = \sigma^2$.

Рассмотрим четыре оценки:
\[
\hat \mu_A = (X_1 + X_2) / 2, \quad \hat\mu_B = (X_1 + X_2 + X_3) / 3,  \hat \mu_C = 2X_1 - X2, 
\quad \hat\mu_D = (X_1 + X_2 + \ldots +  X_{20}) / 21. 
\]

\begin{enumerate}
  \item Какая из приведенных оценок для $\mu$ является несмещённой?
  \item У какой несмещённой оценки самая маленькая дисперсия?
  \item Выберите наиболее эффективную оценку в этом множестве по критерию $MSE$, если $\sigma = 0.5\mu$. 
\end{enumerate}

\item Величины $X_1$ и $X_2$  независимы и равномерны на отрезке $[0;a]$ с неизвестным $a$ и $Y = \min\{ X_1, X_2\}$.

\begin{enumerate}
  \item При каком $\beta$ оценка $\hat a=\beta Y$ для параметра $a$ будет несмещённой?
  \item При каком $\beta$ оценка $\hat a=\beta Y$ для параметра $a$ будет наиболее эффективной по критерию $MSE$?
\end{enumerate}

\item Величины $X_i$ независимы и имеют закон распределения 

\begin{minipage}{\textwidth}
  \begin{tabular}{crrr}
    \toprule
     $x$ & $0$ & $1$ & $a$ \\
    \midrule
    $\P(X = x)$ & $1/4$ & $1/4$ & $2/4$ \\
    \bottomrule
\end{tabular}
\end{minipage}

\begin{enumerate}
    \item Постройте состоятельную оценку для неизвестного $a$.
    \item Возможно ли в этой задаче построить несмещённую оценку для $a$? 
\end{enumerate}


\end{enumerate}



\end{document}
