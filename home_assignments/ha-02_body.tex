\section*{Домашнее задание 2}

% свойства оценок
Дедлайн: 2025-02-23, 23:59.

Оцениваемые задачи:

\begin{enumerate}
    \item Величины $y_1$, $y_2$, \dots, $y_n$ независимы и равномерны отрезке на $[0; a]$ с неизвестным $a > 5$.
    Никола Тесла хочет оценить неизвестный параметр $b = \P(y_i > 5)$.

    Рассмотрим две оценки: $\hat b_n$ — доля наблюдений в выборке, оказавшихся больше $5$ и $\hat b'_n = 1 - 2.5/\bar y$.

    \begin{enumerate}
        \item Является ли оценка $\hat b_n$ несмещённой? состоятельной?
        \item Является ли оценка $\hat b'_n$ несмещённой? состоятельной?
    \end{enumerate}
    

\item Величины $y_i$ независимы и имеют функцию плотности
\[
f(y) = \begin{cases}
3y^2/\theta^3, \text{ если } y\in[0;\theta]; \\
0, \text{ иначе.}
\end{cases}
\]

\begin{enumerate}
\item Найдите оценку $\hat\theta$ неизвестного параметра $\theta$ методом максимального правдоподобия.
\item Является ли оценка $\hat\theta$ несмещённой? 
\item Является ли оценка $\hat\theta$ состоятельной?
\item Найдите функцию плотности оценки $\hat\theta$.
\item На какую величину нужно домножить оценку $\hat\theta$, чтобы она стала несмещённой?
\end{enumerate}

Подсказка: ответ на пункт (б) можно получить без вычислений и интегралов :)

\end{enumerate}

Неоцениваемые задачи в удовольствие:

\begin{enumerate}[resume]
\item Величина $Y$ имеет биномиальное распределение $\dBin(n, p)$.
\begin{enumerate}
\item Является ли оценка $\hat p = Y/n$ для $p$ несмещённой? Если является смещённой, то скорректируйте оценку так, чтобы она стала несмещённой.
\item Чему равна теоретическая дисперсия $\sigma^2$ величины $Y$?
\item Является ли оценка $\hat \sigma^2 = n \hat p (1- \hat p)$ для $\sigma^2$ несмещённой? Если является смещённой, то скорректируйте оценку так, чтобы она стала несмещённой.
\end{enumerate}

\item Величины $X_i$ независимы и одинаково распределены с неизвестными $\E(X_i) = \mu$ и $\Var(X_i) = \sigma^2$.

Рассмотрим четыре оценки:
\[
\hat \mu_A = (X_1 + X_2) / 2, \quad \hat\mu_B = (X_1 + X_2 + X_3) / 3,  \hat \mu_C = 2X_1 - X2, 
\quad \hat\mu_D = (X_1 + X_2 + \ldots +  X_{20}) / 21. 
\]

\begin{enumerate}
  \item Какая из приведенных оценок для $\mu$ является несмещённой?
  \item У какой несмещённой оценки самая маленькая дисперсия?
  \item Выберите наиболее эффективную оценку в этом множестве по критерию $MSE$, если $\sigma = 0.5\mu$. 
\end{enumerate}

\item Величины $X_1$ и $X_2$  независимы и равномерны на отрезке $[0;a]$ с неизвестным $a$ и $Y = \min\{ X_1, X_2\}$.

\begin{enumerate}
  \item При каком $\beta$ оценка $\hat a=\beta Y$ для параметра $a$ будет несмещённой?
  \item При каком $\beta$ оценка $\hat a=\beta Y$ для параметра $a$ будет наиболее эффективной по критерию $MSE$?
\end{enumerate}

\item Величины $X_i$ независимы и имеют закон распределения 

\begin{minipage}{\textwidth}
  \begin{tabular}{crrr}
    \toprule
     $x$ & $0$ & $1$ & $a$ \\
    \midrule
    $\P(X = x)$ & $1/4$ & $1/4$ & $2/4$ \\
    \bottomrule
\end{tabular}
\end{minipage}

\begin{enumerate}
    \item Постройте состоятельную оценку для неизвестного $a$.
    \item Возможно ли в этой задаче построить несмещённую оценку для $a$? 
\end{enumerate}


\end{enumerate}

