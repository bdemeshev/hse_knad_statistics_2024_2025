

\section*{Домашнее задание 6}

% интервальные оценки
Дедлайн: 2025-06-18, 23:59.

Оцениваемые задачи:

\begin{enumerate}
\item Случайные величины $y_1$, \dots, $y_n$ независимы и одинаково распределены с $\P(y_i = 1) = p$ и $\P(y_i = 0) = 1 - p$. 
Рассмотрим бутстрэп-выборку $y_1^*$, \dots, $y_n^*$.
\begin{enumerate}
    \item Найдите $\P(y_1^* = y_1)$ и $\P(y_1^* = y_2^*)$.
    \item Найдите $\E(y_i^*)$ и $\Var(y_i^*)$.
    \item Найдите $\Cov(y_1^*, y_2^*)$.
\end{enumerate}

\item Винни-Пух хочет проверить $5$ нулевых гипотез. 
Он посчитал $p$-значения для каждой из них, $p = (0.03, 0.04, 0.08, 0.15, 0.30)$.
\begin{enumerate}
    \item Какие гипотезы отвергнет алгоритм Холма~— Бонферонни, гарантирующий $FWER = 0.2$?
    \item Какие гипотезы отвергнет алгоритм Беньямини~— Хохберга, гарантирующий $FDR = 0.2$?
\end{enumerate}
     

\end{enumerate}

Неоцениваемые задачи в удовольствие:

\begin{enumerate}[resume]
    \item Величины $y_1$, \dots, $y_n$ независимы и одинаково распределены 
    с функцией плотности 
    \[
    f(y) = \begin{cases}
        \exp(a - y) \text{ если } y \geq a, \\
        0, \text{ иначе.}
    \end{cases}
    \]
    \begin{enumerate}
        \item Найдите оценку неизвестного параметра $a$ методом максимального правдоподобия. 
        \item Какова фактическая вероятность накрытия параметра $a$ при построении наивного бутстрэп доверительного интервала с номинальной вероятностью накрытия $95\%$?
    \end{enumerate}

    \item Опишите алгоритм наивного бутстрэпа для построения 95\% доверительного интервала для истинной медианы распределения. 

    \item У исследователя всего две нулевых гипотезы. 
    Каждая из них априорно верно с вероятностью $0.2$ независимо от других. 
    При верной отдельной нулевой гипотезы $H_j$ распределение соответствующей ей тестовой статистики непрерывно.
    Для упрощения будем считать, что если отдельная нулевая гипотеза $H_j$ не верна, то её $p$-значение в точности равно $0$.

    \begin{enumerate}
        \item Вспомните, как распределено $p$-значение при верной нулевой гипотезе.
        \item Рассмотри алгоритм Холма~— Бонферонни, гарантирующий $FWER = 0.2$.
        Какова условная вероятность того, что он отвергнет конкретную нулевую гипотезу с известным $p$-значением равным $u$?
        \item Рассмотри алгоритм Беньямини~— Хохберга, гарантирующий $FDR = 0.2$.
        Какова условная вероятность того, что он отвергнет конкретную нулевую гипотезу с известным $p$-значением равным $u$?
    \end{enumerate}

    \item В одном из вариантов бутстрэпа (subsampling bootstrap) в бутстрэп выборку попадает $m < n$ наблюдений без повторов из исходной выборки в $n$ наблюдений. 
    Предположим, что исходные $n$ наблюдений $y_1$, $y_2$, \dots, $y_n$ независимы и равномерны $\dUnif[0, 1]$.
    Рассмотрим бутстрэп выборку $y_1^*$, \dots, $y_m^*$.
    \begin{enumerate}
        \item Как распределено $y_i^*$?
        \item Найдите $\Corr(y_1^*, y_2^* \mid y_1, y_2, \dots, y_n)$.
        \item Найдите $\Corr(y_1^*, y_2^*)$.
    \end{enumerate}

 \end{enumerate}

