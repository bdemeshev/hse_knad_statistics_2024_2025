
\section*{Домашнее задание 3}

% интервальные оценки
Дедлайн: 2025-04-27, 23:59.

Оцениваемые задачи:

\begin{enumerate}
\item В одном тропическом лесу водятся удавы и питоны. 
    Длина удавов имеет нормальное распределение $\cN(\mu_X, \sigma^2_X)$. 
    По выборке из 10 удавов оказалось, что $\sum X_i = 20$ метрам, а $\sum X_i^2 = 1000$. 
    Длина питонов имеет нормальное распределение $\cN(\mu_Y, \sigma^2_Y)$. 
    По выборке из 20 питонов оказалось, что $\sum Y_i = 60$ метрам, а $\sum Y_i^2 = 4000$.
    Все наблюдения независимы между собой. 
    \begin{enumerate}
      % \item Анаконда — это питон или удав?
      \item Постройте точечные оценки для $\mu_X$, $\sigma^2_X$, $\mu_Y$, $\sigma^2_Y$.
      \item Постройте двусторонний 95\%-й доверительный интервал для $\sigma^2_X/\sigma^2_Y$.
      \item Проверьте гипотезу $H_0$: $\sigma^2_X = \sigma^2_Y$ против альтернативной $H_1$: $\sigma^2_Y > \sigma^2_X$ на уровне значимости $5\%$.
      Укажите точное $p$-значение.
      \item Постройте примерный двусторонний 95\%-й доверительный интервал для разницы  $\mu_X - \mu_Y$ с помощью статистики Уэлча.
      \item Проверьте гипотезу $H_0$: $\mu_X = \mu_Y$ против альтернативной $H_1$: $\mu_Y > \mu_X$ на уровне значимости $5\%$ с помощью теста Уэлча.
      Укажите точное $p$-значение.
      % \item Важна ли предпосылка о нормальности при решении предыдущих пунктов?
    \end{enumerate}
     
    

\item Априорное распределение параметра $\theta$ является треугольным на отрезке $[0; 40]$ с модой в точке $30$. 
Наблюдаемая величина $X$ — это индикатор того, что $\theta > 20$.
Оказалось, что $X = 1$.
    \begin{enumerate}
        \item Найдите апостериорную плотность $\theta$.
        \item Найдите апостериорное математическое ожидание $\theta$.
        \item Найдите апостериорную медиану $\theta$.
        \item Постройте 94\% байесовский интервал наивысшей плотности для $\theta$.
        \item Постройте 94\% симметричный по вероятности байесовский интервал для $\theta$.
    \end{enumerate}

Определение треугольного распределения можно найти, например, на википедии :)

\end{enumerate}

Неоцениваемые задачи в удовольствие:

\begin{enumerate}[resume]
    \item Величины  $X_1$, \ldots, $X_n$ независимы и одинаково распределены с функцией плотности 
    \[
    f(x) = \begin{cases}
      \theta x^{\theta-1}, \text{ при }  x\in [0;1] \\
      0, \text{ иначе.}
    \end{cases}
    \]
    
    \begin{enumerate}
    \item Оцените значение $\theta$ с помощью метода максимального правдоподобия.
    \item Оцените дисперсию оценки $\hat\theta_{ML}$ метода максимального правдоподобия.
    \item Как примерно распределена $\hat \theta_{ML}$?
    \item Оцените значение $\theta$ с помощью метода моментов.
    \item Оцените дисперсию оценки $\hat\theta_{MM}$ метода моментов.
    \item Как примерно распределена $\hat \theta_{MM}$?
    \end{enumerate}
    

    \item Цыганка Роза ничего не понимает в статистике, 
    но у неё всегда с собой колода из 36 карт.
    
    Помогите цыганке Розе построить точный 95\%-й доверительный интервал для
    неизвестной вероятности $p$ того, что клиента ждёт дальняя дорога и казённый дом.
    
\item Величины $X_1$, \ldots , $X_n$ независимы и одинаково распределены с функцией плотности
  \[
  f(x) = \begin{cases}
    \frac{\theta \exp(-\theta^2 / 2x)}{\sqrt{2\pi x^{3}}} \text{ при } x\in [0;+\infty), \\
    0, \text{ иначе.}
  \end{cases}
  \]

\begin{enumerate}

  \item Найдите оценку параметра $\theta$ методом максимального правдоподобия, 
  если по выборке из 100 наблюдений оказалось $\sum{1/X_i}=12$.
  \item Найдите оценку параметра $\theta$ методом максимального правдоподобия для произвольной выборки.
\item Найдите теоретическую информацию Фишера $I(\theta)$.
\item  Пользуясь данными по выборке постройте оценку $\hat{I}$ для информации Фишера.
\item  Постройте 90\% доверительный интервал для $\theta$.
Подсказка: $\E(1/X_i)=1/\theta^2$, интеграл берется, например, заменой $x=\theta^2a^{-2}$.
\end{enumerate}

\item Величины $X_1$ и $X_2$ независимы и распределены по Пуассону с  интенсивностью $a$. 
Есть две гипотезы, $H_0$: $a=1$ и $H_a$: $a=2$. Мальвина отвергает $H_0$ в том случае, если $X_1 + X_2 \geq 2$. 
Найдите вероятность ошибок первого и второго рода.

\item Величины $Y_1$, \ldots, $Y_n$ независимы и имеют распределение Бернулли с неизвестным $p$,
$\hat p = \bar Y$. 

\begin{enumerate}
  \item Постройте для неизвестного $p$ доверительный интервал Вальда.
  Для этого вспомните про сходимость 
  \[
  \frac{\hat p - p}{\sqrt{\frac{\hat p (1 - \hat p)}{n}}} \distto \cN(0;1)
  \]
  и решите неравенство 
  \[
    -\zcr \leq \frac{\hat p - p}{\sqrt{\frac{\hat p (1 - \hat p)}{n}}} \leq \zcr.
  \]

  \item Постройте для неизвестного $p$ доверительный интервал Вильсона.
  Для этого воспользуйтесь сходимостью
  \[
  \frac{\hat p - p}{\sqrt{\frac{p (1 - p)}{n}}} \distto \cN(0;1).
  \]
  На этот раз потребуется решить (о ужас!) квадратное неравенство. 
\end{enumerate}
Обозначим центр интервала Вильсона с помощью $\hat p_w$.
\begin{enumerate}[resume]
  \item Докажите, что центр интервала Вильсона $\hat p_w$ можно представить как 
  средневзвешенное классической оценки $\hat p$ и тривиальной оценки $1/2$,
  \[
  \hat p_w = u \hat p + (1 - u) (1/2).  
  \]
  Найдите веса $u$ и $(1-u)$.
  \item Докажите, что центр интервала Вильсона $\hat p_w$ можно
  проинтерпретировать следующим образом: добавим $f$ вымышленных единиц и $f$ вымышленных нулей 
  в выборку и посчитаем классическую оценку вероятности для выборки с вымышленными наблюдениями,
  \[
  \hat p_w = \frac{\sum_{i=1}^n Y_i + f}{n + 2 f}.
  \]
  Какому целому числу примерно равно $f$ для 95\%-го доверительного интервала?
  \item Докажите, что интервал Вильсона можно записать в виде
  \[
    \hat p_w \pm \zcr \cdot \sqrt{\frac{u \hat p (1 - \hat p) + (1 - u) (1/2)^2 }{n_w}}.
  \]
  Найдите $n_w$, а также веса $u$ и $(1 - u)$.

  Таким образом, интервал Вильсона слегка корректирует число наблюдений
  и использует в качестве оценки дисперсии $Y_i$ средневзвешенное между классической оценкой $\hat p (1 - \hat p)$
  и тривиальной оценкой $1/4$.

\end{enumerate}

Доверительный интервал Агрести~— Коулла для уровня доверия 95\% строится следующим образом.
В выборку мысленно добавляют два наблюдения равных единице и два наблюдений равных нулю,
считают оценку доли
\[
\hat p_{ac} = \frac{\sum_{i=1}^n Y_i + 2}{n + 4},  
\] 
а затем строят классический интервал Вальда, используя $\hat p_{ac}$ вместо классической $\hat p$. 

  
\begin{enumerate}[resume]
  \item Правда ли, что при уровне доверия 95\% центры интервала Агрести~— Коулла и Вильсона совпадают?
  \item Какой 95\%-й интервал шире, Агрести~— Коулла или Вильсона?
  \item С помощью симуляций на компьютере сравните фактическую вероятность накрытия 
  неизвестного параметра $p$ интервалами Вальда, Вильсона и Агрести~— Коулла
  c номинальной 95\%-й вероятностью. 
  Для экспериментов возьмите $n=50$ и различные $p$ от 0 до 1 с шагом $0.1$.

\end{enumerate}


\end{enumerate}

