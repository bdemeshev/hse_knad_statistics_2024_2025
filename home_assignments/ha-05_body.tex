

\section*{Домашнее задание 5}

% интервальные оценки
Дедлайн: 2025-06-01, 23:59.

Оцениваемые задачи:

\begin{enumerate}
\item Величины $(y_i)$ независимы и экспоненциально распределены с интенсивностью $\lambda$.

Количество наблюдений $n$ велико. 
Тестируем гипотезу $H_0$: $\lambda = 2$ против альтернативы $\lambda \neq 2$.
\begin{enumerate}
    \item Выведите формулы для теста отношения правдоподобия $LR$, теста множителей Лагранжа $LM$ и теста Вальда $W$.
    \item Проведите тесты для конкретной выборки с $n = 1000$, $\bar y = 2.2$ и уровня значимости 1\%.
\end{enumerate}
     
\item Величины $(y_i)$ независимы и нормально распределены $\cN(\mu, 1)$.

Количество наблюдений $n$ велико. 
Тестируем гипотезу $H_0$: $\mu = 0$ против альтернативы $\mu \neq 0$.
\begin{enumerate}
    \item Выведите формулы для теста отношения правдоподобия $LR$, теста множителей Лагранжа $LM$ и теста Вальда $W$.
    \item Проведите тесты для конкретной выборки с $n = 1000$, $\sum y_i = 1000$, $\sum y_i^2 = 4000$ и уровня значимости 1\%.
\end{enumerate}

\end{enumerate}

Неоцениваемые задачи в удовольствие:

\begin{enumerate}[resume]
    \item Гипотеза $H_0$ описывается $5$-ю независимыми уравнениями, неограниченный максимум лог-правдоподобия равен $\ell_{UR} = -200$, а ограниченный — $\ell_R=-209$.
    Число наблюдений $n$ велико. Альтернативная гипотеза состоит в том, что хотя бы одно уравнение не выполнено.

    \begin{enumerate}
        \item Отвергается ли $H_0$ на уровне значимости $1\%$?
        \item Найдите $p$-значение. 
    \end{enumerate}
    
    \item Оценка неизвестного вектора параметров $a = (a_1, a_2, a_3)$  равна $\hat a = (1, 2, 3)$ с оценкой ковариационной матрицы
    \[
    \widehat{\Var}(\hat a) = \begin{pmatrix}
        9 & -1 & 2 \\
         & 16 & -1 \\
         & & 10 \\
    \end{pmatrix}.
    \]
    Число наблюдений велико.
    Рассмотрим гипотезу $H_0$: $a_1 = a_2 = a_3$ против альтернативы о том, что хотя бы одно уравнение не выполнено.
    \begin{enumerate}
        \item Предложите естественную оценку $\hat b$ для вектора $b = (a_1 - a_2, a_2 - a_3)$.
        \item Оцените ковариационную матрицу $\widehat{\Var}(\hat b)$.
        \item Переформулируйте $H_0$ в терминах вектора $b$.
        \item Проведите тест Вальда гипотезы $H_0$ на уровне значимости $5\%$.
    \end{enumerate}
    
    \item Мы оцениваем три неизвестных параметра, $(\theta_1, \theta_2, \theta_3)$. 
    При максимизации с учётом ограничений гипотезы $H_0$ оказывается, что градиент лог-правдоподобия равен $\grad \ell = (-0.1, 0.2, 0)$,
    а матрица Гессе в точке ограниченного экстремума равна 
    \[
    H = \begin{pmatrix}
        -5 & -2 & 0 \\
         & -6 & 0 \\
         & & -10
    \end{pmatrix}
    \]
    Число наблюдений велико. 

    \begin{enumerate}
        \item Чему равен градиент лог-правдоподобия в точке неограниченного экстремума?
        \item Протестируйте $H_0$ на уровне значимости 1\% с помощью теста множителей Лагранжа.
    \end{enumerate}

    \item Вспомним классический хи-квадрат тест Пирсона на соответствие выборки заданному дискретному закону распределения со 
    статистикой 
    \[
    S = \sum_{i=1}^k \frac{(f_i - np_i)^2}{np_i},
    \]
    где $k$ — число клеток таблицы, $f_i$ — количество наблюдений, попавших в $i$-ую клетку таблицы, а $p_1$, $p_2$, \dots, $p_k$ — вероятности, предполагаемые в $H_0$.
    
    С каким тестом ($LR$/$LM$/$W$) совпадает данная статистика?
    
    
    \item 

    \item \end{enumerate}

