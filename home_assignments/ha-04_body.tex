
\section*{Домашнее задание 4}

% интервальные оценки
Дедлайн: 2025-05-18, 23:59.

Оцениваемые задачи:

\begin{enumerate}
\item Величины $(y_i)$ независимы и одинаково непрерывно распределены. 
Всего есть $1000$ наблюдений. 
Постройте 95\%-й интервал для 90\%-го квантиля с помощью выборочных квантилей. 

Если для вычисления необходимых выборочных квантилей использовался код, то приведите его. 
     
\item Есть две выборки: $x = (2.7, 3.5, 4.2, 6.7)$ и $y = (1.6, 2.9, 3.9)$.
Все наблюдения независимы. 
Величины $(x_i)$ одинаково непрерывно распределены между собой, величины $(y_i)$ одинаково непрерывно распределены между собой. 
Проверьте гипотезу $H_0$ об одинаковом законе распределения в двух выборках, против альтернативной $\P(x_i > y_j) > 0.5$
на уровне значимости $5\%$.

\begin{enumerate}
    \item Проведите тест Манна~— Уитни, используя точное распределение статистики.
    \item Проведите тест Манна~— Уитни, используя нормальную аппроксимацию. Укажите $p$-значение.
\end{enumerate}

\end{enumerate}

Неоцениваемые задачи в удовольствие:

\begin{enumerate}[resume]
    \item Рассмотрим тест знаковых рангов Уилкоксона и связанные пары наблюдений $(x_i, y_i)$.
    При верной $H_0$ разницы $D_i = x_i - y_i$ одинаково непрерывно распределены и независимы. 

    Рассмотрим сумму знаковых рангов $WSR = \sum_{i=1}^n \sign(D_i) \rank (\abs{D_i})$.

    Найдите ожидание $\E(WSR)$ и дисперсию $\Var(WSR)$ при верной $H_0$.

    \item Величины $(X_i)$ независимы и одинаково распределены с неизвестными $\E(X_i) = \mu$ и $\Var(X_i) = \sigma^2$.
    По выборке из $1000$ наблюдений оказалось, что $\bar X = 30$, а несмещённая выборочная дисперсия равна $900$. 
    \begin{enumerate}
        \item Постройте асимптотический 95\%-й доверительный интервал для $\mu$. 
        Укажите $p$-значение для гипотезы $H_0$: $\mu = 35$ против альтернативной $H_a$: $\mu \neq 35$.
        \item Постройте асимптотический 95\%-й предсказательный интервал для $X_{1001}$.
        \item Постройте асимптотический 95\%-й предсказательный интервал для $(X_{1001} + X_{1002})/2$.
    \end{enumerate}

    \item Бариста Борис заметил, что в последнее время посетители заказывают только капуччино и раф. 
    Предположим, что посетители выбирают напиток независимо друг от друга, 
    а вероятность выбора капуччино постоянна и равна неизвестному числу $p$.
    
    У Бориса есть только две гипотезы, $H_0: p =1/3$ и $H_a: p =2/3$, 
    в которые он до получения данных верит с вероятностями $0.6$ и $0.4$, соответственно.
    
    Из первых 100 утренних посетителей $S = 40$ выбрали капуччино. 
    Борис хочет измерить разными способами, насколько этот наблюдаемый результат соотносится с гипотезами.
    
    \begin{enumerate}
      \item Найдите $\P(H_0 \mid S=40)$ и $\P(H_a \mid S=40)$.
    \end{enumerate}
    
    Борис решил на следующий день повторить эксперимент и снова посчитать $S_{\text{new}}$, 
    количество клиентов из первых ста, которые выберут капуччино. 
    \begin{enumerate}[resume]
      \item Найдите $\P(S_{\text{new}} \geq S \mid S=40, H_0)$ и $\P(S_{\text{new}} \geq S \mid S=40, H_a)$.
    \end{enumerate}
    
    \begin{enumerate}[resume]
      \item Какие из вероятностей можно посчитать без мнения Бориса о $\P(H_0)$ и $\P(H_a)$?
      \item Какая из вероятностей называется $p$-значением\index{$p$-значение} для гипотезы $H_0$ и статистики $S$?
    \end{enumerate}
    
    
    \item По таблице сопряжённости проверьте гипотезу о независимости двух признаков на уровне значимости 5\%
    против альтернативной гипотезы о зависимости признаков. Укажите $p$-значение.
    
    \begin{tabular}{*{3}{c}}
        \toprule
         & $X = A$ & $X = B$ \\
        \midrule
        $Y = C$ & $50$ & $60$ \\
        $Y = D$ & $20$ & $30$ \\
        $Y = E$ & $60$ & $50$ \\
        \bottomrule
    \end{tabular}
        
    \item Рассмотрим таблицу сопряжённости
    
    \begin{tabular}{*{4}{c}}
        \toprule
        $X = A$ & $X = B$ & $X = C$ & $X = D$ \\
        \midrule
        $50$ & $70$ & $80$ & $60$ \\
        \bottomrule
    \end{tabular}
    
    \begin{enumerate}
        \item На уровне значимости 5\% проверьте гипотезу об одинаковых вероятностях $p_a = p_b = p_c = p_d$ против альтернативной о том, что хотя бы одно из равенств нарушено.
        \item На уровне значимости 5\% проверьте гипотезу об одинаковых вероятностях $p_a = p_b = p_c = p_d$ против альтернативной о том, что $p_a \neq p_b = p_c$.
        \item На уровне значимости 5\% проверьте гипотезу об одинаковых вероятностях $p_a = p_b = p_c$ против альтернативной о том, что $p_a \neq p_b = p_c$.
    \end{enumerate}

    В каждом случае укажите $p$-значение.
\end{enumerate}

