
\documentclass[12pt]{article}

% \usepackage{physics}

\usepackage{hyperref}
\hypersetup{
    colorlinks=true,
    linkcolor=blue,
    filecolor=magenta,      
    urlcolor=cyan,
    pdftitle={Overleaf Example},
    pdfpagemode=FullScreen,
    }

\usepackage{tikzducks}

\usepackage{tikz} % картинки в tikz
\usepackage{microtype} % свешивание пунктуации

\usepackage{array} % для столбцов фиксированной ширины

\usepackage{indentfirst} % отступ в первом параграфе

\usepackage{sectsty} % для центрирования названий частей
\allsectionsfont{\centering}

\usepackage{amsmath, amsfonts, amssymb} % куча стандартных математических плюшек

\usepackage{comment}

\usepackage[top=2cm, left=1.2cm, right=1.2cm, bottom=2cm]{geometry} % размер текста на странице

\usepackage{lastpage} % чтобы узнать номер последней страницы

\usepackage{enumitem} % дополнительные плюшки для списков
%  например \begin{enumerate}[resume] позволяет продолжить нумерацию в новом списке
\usepackage{caption}

\usepackage{url} % to use \url{link to web}


\newcommand{\smallduck}{\begin{tikzpicture}[scale=0.3]
    \duck[
        cape=black,
        hat=black,
        mask=black
    ]
    \end{tikzpicture}}

\usepackage{fancyhdr} % весёлые колонтитулы
\pagestyle{fancy}
\lhead{}
\chead{}
\rhead{Домашние задания для самураев}
\lfoot{}
\cfoot{}
\rfoot{}

\renewcommand{\headrulewidth}{0.4pt}
\renewcommand{\footrulewidth}{0.4pt}

\usepackage{tcolorbox} % рамочки!

\usepackage{todonotes} % для вставки в документ заметок о том, что осталось сделать
% \todo{Здесь надо коэффициенты исправить}
% \missingfigure{Здесь будет Последний день Помпеи}
% \listoftodos - печатает все поставленные \todo'шки


% более красивые таблицы
\usepackage{booktabs}
% заповеди из докупентации:
% 1. Не используйте вертикальные линни
% 2. Не используйте двойные линии
% 3. Единицы измерения - в шапку таблицы
% 4. Не сокращайте .1 вместо 0.1
% 5. Повторяющееся значение повторяйте, а не говорите "то же"


\setcounter{MaxMatrixCols}{20}
% by crazy default pmatrix supports only 10 cols :)


\usepackage{fontspec}
\usepackage{libertine}
\usepackage{polyglossia}

\setmainlanguage{russian}
\setotherlanguages{english}

% download "Linux Libertine" fonts:
% http://www.linuxlibertine.org/index.php?id=91&L=1
% \setmainfont{Linux Libertine O} % or Helvetica, Arial, Cambria
% why do we need \newfontfamily:
% http://tex.stackexchange.com/questions/91507/
% \newfontfamily{\cyrillicfonttt}{Linux Libertine O}

\AddEnumerateCounter{\asbuk}{\russian@alph}{щ} % для списков с русскими буквами
\setlist[enumerate, 2]{label=\asbuk*),ref=\asbuk*}

%% эконометрические сокращения
\DeclareMathOperator{\pCorr}{\mathrm{pCorr}}
\DeclareMathOperator{\Cov}{\mathbb{C}ov}
\DeclareMathOperator{\Corr}{\mathbb{C}orr}
\DeclareMathOperator{\Var}{\mathbb{V}ar}
\DeclareMathOperator{\col}{col}
\DeclareMathOperator{\row}{row}

\let\P\relax
\DeclareMathOperator{\P}{\mathbb{P}}

\let\H\relax
\DeclareMathOperator{\H}{\mathbb{H}}

\DeclareMathOperator{\CE}{\mathrm{CE}}



\DeclareMathOperator{\E}{\mathbb{E}}
% \DeclareMathOperator{\tr}{trace}
\DeclareMathOperator{\card}{card}

\DeclareMathOperator{\Convex}{Convex}

\newcommand \cN{\mathcal{N}}
\newcommand \dN{\mathcal{N}}
\newcommand \dBin{\mathrm{Bin}}
\newcommand{\dUnif}{\mathrm{Unif}}


\newcommand \RR{\mathbb{R}}
\newcommand \NN{\mathbb{N}}





\begin{document}



\section*{Домашнее задание 6}

% интервальные оценки
Дедлайн: 2025-06-18, 23:59.

Оцениваемые задачи:

\begin{enumerate}
\item Случайные величины $y_1$, \dots, $y_n$ независимы и одинаково распределены с $\P(y_i = 1) = p$ и $\P(y_i = 0) = 1 - p$. 
Рассмотрим бутстрэп-выборку $y_1^*$, \dots, $y_n^*$.
\begin{enumerate}
    \item Найдите $\P(y_1^* = y_1)$ и $\P(y_1^* = y_2^*)$.
    \item Найдите $\E(y_i^*)$ и $\Var(y_i^*)$.
    \item Найдите $\Cov(y_1^*, y_2^*)$.
\end{enumerate}

\item Винни-Пух хочет проверить $5$ нулевых гипотез. 
Он посчитал $p$-значения для каждой из них, $p = (0.03, 0.04, 0.08, 0.15, 0.30)$.
\begin{enumerate}
    \item Какие гипотезы отвергнет алгоритм Холма~— Бонферонни, гарантирующий $FWER = 0.2$?
    \item Какие гипотезы отвергнет алгоритм Беньямини~— Хохберга, гарантирующий $FDR = 0.2$?
\end{enumerate}
     

\end{enumerate}

Неоцениваемые задачи в удовольствие:

\begin{enumerate}[resume]
    \item Величины $y_1$, \dots, $y_n$ независимы и одинаково распределены 
    с функцией плотности 
    \[
    f(y) = \begin{cases}
        \exp(a - y) \text{ если } y \geq a, \\
        0, \text{ иначе.}
    \end{cases}
    \]
    \begin{enumerate}
        \item Найдите оценку неизвестного параметра $a$ методом максимального правдоподобия. 
        \item Какова фактическая вероятность накрытия параметра $a$ при построении наивного бутстрэп доверительного интервала с номинальной вероятностью накрытия $95\%$?
    \end{enumerate}

    \item Опишите алгоритм наивного бутстрэпа для построения 95\% доверительного интервала для истинной медианы распределения. 

    \item У исследователя всего две нулевых гипотезы. 
    Каждая из них априорно верно с вероятностью $0.2$ независимо от других. 
    При верной отдельной нулевой гипотезы $H_j$ распределение соответствующей ей тестовой статистики непрерывно.
    Для упрощения будем считать, что если отдельная нулевая гипотеза $H_j$ не верна, то её $p$-значение в точности равно $0$.

    \begin{enumerate}
        \item Вспомните, как распределено $p$-значение при верной нулевой гипотезе.
        \item Рассмотри алгоритм Холма~— Бонферонни, гарантирующий $FWER = 0.2$.
        Какова условная вероятность того, что он отвергнет конкретную нулевую гипотезу с известным $p$-значением равным $u$?
        \item Рассмотри алгоритм Беньямини~— Хохберга, гарантирующий $FDR = 0.2$.
        Какова условная вероятность того, что он отвергнет конкретную нулевую гипотезу с известным $p$-значением равным $u$?
    \end{enumerate}

    \item В одном из вариантов бутстрэпа (subsampling bootstrap) в бутстрэп выборку попадает $m < n$ наблюдений без повторов из исходной выборки в $n$ наблюдений. 
    Предположим, что исходные $n$ наблюдений $y_1$, $y_2$, \dots, $y_n$ независимы и равномерны $\dUnif[0, 1]$.
    Рассмотрим бутстрэп выборку $y_1^*$, \dots, $y_m^*$.
    \begin{enumerate}
        \item Как распределено $y_i^*$?
        \item Найдите $\Corr(y_1^*, y_2^* \mid y_1, y_2, \dots, y_n)$.
        \item Найдите $\Corr(y_1^*, y_2^*)$.
    \end{enumerate}

 \end{enumerate}



\end{document}
