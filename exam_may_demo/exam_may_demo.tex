% arara: xelatex
\documentclass[12pt]{article}

% \usepackage{physics}

\usepackage{hyperref}
\hypersetup{
    colorlinks=true,
    linkcolor=blue,
    filecolor=magenta,      
    urlcolor=cyan,
    pdftitle={Overleaf Example},
    pdfpagemode=FullScreen,
    }
\urlstyle{same}

\usepackage{tikzducks}

\usepackage{tikz} % картинки в tikz
\usepackage{microtype} % свешивание пунктуации

\usepackage{array} % для столбцов фиксированной ширины

\usepackage{indentfirst} % отступ в первом параграфе

\usepackage{sectsty} % для центрирования названий частей
\allsectionsfont{\centering}

\usepackage{amsmath, amsfonts, amssymb} % куча стандартных математических плюшек

\usepackage{mathtools}
\usepackage{comment}

\usepackage[top=2cm, left=1.2cm, right=1.2cm, bottom=2cm]{geometry} % размер текста на странице

\usepackage{lastpage} % чтобы узнать номер последней страницы

\usepackage{enumitem} % дополнительные плюшки для списков
%  например \begin{enumerate}[resume] позволяет продолжить нумерацию в новом списке
\usepackage{caption}

\usepackage{url} % to use \url{link to web}


\newcommand{\smallduck}{\begin{tikzpicture}[scale=0.3]
    \duck[
        cape=black,
        hat=black,
        mask=black
    ]
    \end{tikzpicture}}

\usepackage{fancyhdr} % весёлые колонтитулы
\pagestyle{fancy}
\lhead{}
\chead{}
\rhead{Демо версия, май 2025}
\lfoot{}
\cfoot{}
\rfoot{}

\renewcommand{\headrulewidth}{0.4pt}
\renewcommand{\footrulewidth}{0.4pt}

\usepackage{tcolorbox} % рамочки!

\usepackage{todonotes} % для вставки в документ заметок о том, что осталось сделать
% \todo{Здесь надо коэффициенты исправить}
% \missingfigure{Здесь будет Последний день Помпеи}
% \listoftodos - печатает все поставленные \todo'шки


% более красивые таблицы
\usepackage{booktabs}
% заповеди из докупентации:
% 1. Не используйте вертикальные линни
% 2. Не используйте двойные линии
% 3. Единицы измерения - в шапку таблицы
% 4. Не сокращайте .1 вместо 0.1
% 5. Повторяющееся значение повторяйте, а не говорите "то же"


\setcounter{MaxMatrixCols}{20}
% by crazy default pmatrix supports only 10 cols :)


\usepackage{fontspec}
\usepackage{libertine}
\usepackage{polyglossia}

\setmainlanguage{russian}
\setotherlanguages{english}

% download "Linux Libertine" fonts:
% http://www.linuxlibertine.org/index.php?id=91&L=1
% \setmainfont{Linux Libertine O} % or Helvetica, Arial, Cambria
% why do we need \newfontfamily:
% http://tex.stackexchange.com/questions/91507/
% \newfontfamily{\cyrillicfonttt}{Linux Libertine O}

\AddEnumerateCounter{\asbuk}{\russian@alph}{щ} % для списков с русскими буквами
\setlist[enumerate, 2]{label=\asbuk*),ref=\asbuk*}

%% эконометрические сокращения
\DeclareMathOperator{\Cov}{\mathbb{C}ov}
\DeclareMathOperator{\Corr}{\mathbb{C}orr}
\DeclareMathOperator{\Var}{\mathbb{V}ar}
\DeclareMathOperator{\pCorr}{\mathrm{pCorr}}
\DeclareMathOperator{\col}{col}
\DeclareMathOperator{\row}{row}

\let\P\relax
\DeclareMathOperator{\P}{\mathbb{P}}

\DeclarePairedDelimiter{\abs}{\lvert}{\rvert}
\DeclarePairedDelimiter{\scalp}{\langle}{\rangle}

\let\H\relax
\DeclareMathOperator{\H}{\mathbb{H}}
\DeclareMathOperator{\plim}{plim}

\DeclareMathOperator{\E}{\mathbb{E}}
% \DeclareMathOperator{\tr}{trace}
\DeclareMathOperator{\card}{card}

\DeclareMathOperator{\Convex}{Convex}

\newcommand \cN{\mathcal{N}}
\newcommand \dN{\mathcal{N}}


\newcommand \RR{\mathbb{R}}
\newcommand \NN{\mathbb{N}}

\newcommand{\dBern}{\mathrm{Bern}}
\newcommand{\dBin}{\mathrm{Bin}}
\newcommand{\dGamma}{\mathrm{Gamma}}
\newcommand{\dBeta}{\mathrm{Beta}}



\begin{document}

\section*{Формат}

В контрольной будет 6 задач.
Задачи имеют равный вес. 
Продолжительность работы 120 минут. 
Можно будет использовать в качестве разрешенной шпаргалки один лист А4 со всех шести его сторон.



\section*{Вариант «Васко да Гама»}
\begin{enumerate}
    \item В пруду встречаются караси, щуки и налимы. 
    Вид каждой выловленной рыбы не зависит от других выловленных рыб.
    Кот Матроскин поймал 200 рыб: 50 карасей, 70 щук и 80 налимов. 

    \begin{enumerate}
        \item С помощью критерия хи-квадрат Пирсона на уровне значимости $5\%$ проверьте гипотезу о том, что вероятности равны $0.2$, $0.4$ и $0.4$, 
        соответственно, против альтернативной гипотезы о том, что хотя бы одна из вероятностей отлична от предлагаемых.
        \item Укажите $p$-значение для теста из пункта (а).
    \end{enumerate}
    
    \item Дональд Трамп проверяет $100$ нулевых гипотез против $100$ альтернативных гипотез с помощью подсчёта $p$-значений.
    Все нулевые гипотезы верны. 
    Предположим, что $p$-значения независимы. 
    Дональд отвергает очередную $H_0$, если её $p$-значение меньше $0.02$.

    \begin{enumerate}
        \item Какова вероятность того, что Дональд ошибочно отвергнет хотя бы одну нулевую гипотезу?
        \item Какова вероятность того, что Дональд ошибочно отвергнет ровно $2$ нулевых гипотезы?
        \item Сколько в среднем нулевых гипотез отвергает Дональд?
    \end{enumerate}
    
    
    \item Наблюдаемая величина $X$ распределена равномерно на отрезке $[0, a]$, где $a$ — неизвестный параметр.
    
    Постройте 95\%-й интервал вида $[0, kX]$ для 80\%-го квантиля величины $X$. 

    \item Величины $(x_i)$ независимы и одинаково распределены с неизвестным ожиданием $\mu$ и конечной дисперсией.
    По выборке размера $n = 1000$ оказалось, что выборочное среднее равно $20$, 
    а несмещённая оценка дисперсии $x_i$ равна $500$.

    Винни-Пух хочет протестировать гипотезу $H_0$: $\mu = 15$ против альтернативы $H_1$: $\mu > 15$ на уровне значимости 5\%.
    \begin{enumerate}
        \item Проведите данные тест с помощью сравнения критического и наблюдаемого значения классической статистики. 
        \item Укажите $p$-значение для теста из пункта (a).
    \end{enumerate}

    \item Известны результаты шести студентов по первой контрольной работе $x = (43, 56, 59, 48, 29, 63)$ и 
    результаты тех же студентов по второй работе $y = (48, 50, 66, 40, 39, 59)$.

    Предположим, что разницы результатов независимы и одинаково непрерывно симметрично распределены с ожиданием $\mu$.

    С помощью теста знаковых рангов Уилкоксона проверьте гипотезу $\mu = 0$ против альтернативы $\mu \neq 0$ на уровне значимости $5\%$.


    \item Априорное распределение параметра $a$ является равномерным на отрезке $[0, 10]$.
    Ненаблюдаемая величина $Y$ равномерно распределены на отрезке $[0, 20]$.
    Известно, что $Y$ оказалось больше $a$.

    \begin{enumerate}
        \item Найдите апостериорное распределение параметра $a$. 
        \item Найдите апостериорное ожидание и медиану параметра $a$. 
        \item Постройте любой 94\% апостериорный интервал для $a$. 
    \end{enumerate}

\end{enumerate}

\end{document}



\end{document}

